%%
% Plantilla de Memoria
% Modificación de una plantilla de Latex de Nicolas Diaz para adaptarla 
% al castellano y a las necesidades de escribir informática y matemáticas.
%
% Editada por: Mario Román
%
% License:
% CC BY-NC-SA 3.0 (http://creativecommons.org/licenses/by-nc-sa/3.0/)
%%

%%%%%%%%%%%%%%%%%%%%%
% Thin Sectioned Essay
% LaTeX Template
% Version 1.0 (3/8/13)
%
% This template has been downloaded from:
% http://www.LaTeXTemplates.com
%
% Original Author:
% Nicolas Diaz (nsdiaz@uc.cl) with extensive modifications by:
% Vel (vel@latextemplates.com)
%
% License:
% CC BY-NC-SA 3.0 (http://creativecommons.org/licenses/by-nc-sa/3.0/)
%
%%%%%%%%%%%%%%%%%%%%%

%----------------------------------------------------------------------------------------
%	PAQUETES Y CONFIGURACIÓN DEL DOCUMENTO
%----------------------------------------------------------------------------------------

%% Configuración del papel.
% microtype: Tipografía.
% mathpazo: Usa la fuente Palatino.
\documentclass[a4paper, 11pt]{article}
\usepackage[protrusion=true,expansion=true]{microtype}
\usepackage{mathpazo}

% Indentación de párrafos para Palatino
\setlength{\parindent}{0pt}
  \parskip=8pt
\linespread{1.05} % Change line spacing here, Palatino benefits from a slight increase by default


%% Castellano.
% noquoting: Permite uso de comillas no españolas.
% lcroman: Permite la enumeración con numerales romanos en minúscula.
% fontenc: Usa la fuente completa para que pueda copiarse correctamente del pdf.
\usepackage[spanish,es-noquoting,es-lcroman]{babel}
\usepackage[utf8]{inputenc}
\usepackage[T1]{fontenc}
\selectlanguage{spanish}

%% Gráficos
\usepackage{graphicx} % Required for including pictures
\usepackage{wrapfig} % Allows in-line images
\usepackage[usenames,dvipsnames]{color} % Coloring code

%% Matemáticas
\usepackage{amsmath}

\makeatletter

% Hipervínculos
\usepackage[hidelinks]{hyperref}

%% Para incluir archivos en texto plano
\usepackage{listings}

%----------------------------------------------------------------------------------------
%	TÍTULO
%----------------------------------------------------------------------------------------
% Configuraciones para el título.
% El título no debe editarse aquí.
\renewcommand{\maketitle}{
  \begin{flushright} % Right align
  
  {\LARGE\@title} % Increase the font size of the title
  
  \vspace{50pt} % Some vertical space between the title and author name
  
  {\large\@author} % Author name
  \\\@date % Date
  \vspace{40pt} % Some vertical space between the author block and abstract
  \end{flushright}
}

% Título
\title{\textbf{Seguridad y Protección de Sistemas Informáticos (SPSI)}\\ % Title
Criptosistemas Asimétricos} % Subtitle

\author{\textsc{Óscar Bermúdez Garrido} % Author
\\{\textit{Universidad de Granada}}} % Institution

\date{\today} % Date


%----------------------------------------------------------------------------------------
%	DOCUMENTO
%----------------------------------------------------------------------------------------

\begin{document}

\maketitle % Print the title section

% Resumen (Descomentar para usarlo)
\renewcommand{\abstractname}{Resumen} % Uncomment to change the name of the abstract to something else

% Índice
{\parskip=2pt
  \tableofcontents
}
\pagebreak

%% Inicio del documento

\section{Creación de los archivos}
	\subsection{Input.bin}
		Al igual que en la práctica anterior, input.bin se crea usando \href{http://manpages.ubuntu.com/manpages/zesty/en/man1/dd.1.html}
		{\textit{dd}} con entrada \textit{/dev/zero}.
		
	\subsection{Claves RSA}
		Para generar las claves RSA, nos basta con usar \href{http://manpages.ubuntu.com/manpages/zesty/en/man1/openssl.1ssl.html}
		{\textit{openssl}} con la opción \href{https://www.openssl.org/docs/man1.0.2/apps/genrsa.html}{\textit{genrsa}} para
		generar una clave del tamaño que queramos, y con \href{https://www.openssl.org/docs/man1.0.2/apps/rsa.html}{\textit{rsa}}
		para trabajar con ella, por ejemplo, para cifrarla o para extraer la clave pública de la privada:
		\begin{small}
			\verb|openssl genrsa -out $name"RSAkey.pem" $key_size| \\
			\verb|openssl rsa -aes128 -passout pass:$passwd -in $name"RSAkey.pem" -out $name"RSApriv.pem"|
			\verb|openssl rsa -pubout -in $name"RSAkey.pem" -out $name"RSApub.pem"|
		\end{small}
		
		Como especial mención, podemos pasar una clave como parámetro en un fichero (\textit{file: <archivo>}) o una variable
		(\textit{pass: \$<variable>}) usando uso de los parámetros \textit{passin} o \textit{passout} dependiendo si la clave
		va para acceder al fichero de entrada o de salida.
		
	\subsection{sessionkey}
	    Este archivo, lo crearemos usando el comando \href{https://www.openssl.org/docs/man1.0.2/apps/rand.html}{\textit{rand}} de \href{http://manpages.ubuntu.com/manpages/zesty/en/man1/openssl.1ssl.html}{\textit{openssl}} para generar los caracteres
	    hexadecimales que necesitemos usando \textit{hex}. En mi caso, voy a usar \textit{aes-256-cbc}, así que necesitaré 32
	    caracteres hexadecimales para la clave. Además, también hay que añadir el modo de cifrado después. En resumen, el archivo
	    se generaría así:
		
		\begin{small}
			\verb|openssl rand 32 -hex -out sessionkey| \\
			\verb|echo $mode >> sessionkey|
		\end{small}
		
	\subsection{Claves de Curvas Elípticas}
		Lo primero que debemos hacer es encontrar la curva que queremos usar. Para ello, me parece muy útil ejecutar un
		bucle que recorra todas las curvas de las que dispone nuestro programa y nos muestre sus valores. Para conocer
		de qué curvas dispone, haremos uso de \href{https://www.openssl.org/docs/man1.0.2/apps/ecparam.html}
		{\textit{ecparam}} junto con el flag \textit{-list\_curves} para listar, y con \textit{-name} y \textit{-param\_enc}
		para mostrar los valores. En particular, sabiendo que nuestras curvas se llamaban \textbf{P-192} y \textbf{B-163},
		sólo recorreremos las curvas con dicho número en el nombre o descripción:
		\begin{verbatim}
		for i in $(openssl ecparam -list_curves | grep '163\|192' | cut -d ":" -f 1);
			do
				echo -e "------------------------------------------------------------------\n$i:";
				openssl ecparam -name $i -param_enc explicit -text -noout;
			done
		\end{verbatim}
		
		Tras mostrar esta lista, miraremos nosotros mismos qué curva tiene los parámetros vistos en clase, es decir,
		miramos el primo, el valor de b y el orden. Tras hacer esto, concluimos que \textbf{P-192} se llama
		\textbf{prime192v1} mientras que la curva \textbf{B-163} se llama \textbf{sect163r2}. En mi caso, me quedaré
		con \textbf{P-192}.
		
		Una vez conocido el nombre, basta con almacenarla en un archivo y generar la clave:		
		\begin{small}
			\verb|openssl ecparam -name $curve -out "stdECparam.pem"| \\
			\verb|openssl ecparam -in "stdECparam.pem" -genkey -out $name"ECkey.pem"|
		\end{small}
		
		Y, análogamente a como hicimos con las claves RSA, usaremos \href{https://www.openssl.org/docs/man1.0.2/apps/ec.html}
		{\textit{ec}} para trabajar con la clave: \\
		\begin{small}
			\verb|openssl ec -des3 -passout pass:$passwd -in $name"ECkey.pem" -out $name"ECpriv.pem"| \\
			\verb|openssl ec -pubout -in $name"ECkey.pem" -out $name"ECpub.pem"|
		\end{small}
	
	\subsection{Archivos de valores}
		El volcado de los datos en archivos para su posterior análisis que, por sencillez, se llamarán de igual forma que los
		archivos originales pero cambiando su extensión por \textit{.txt} se realiza mediante la instrucción: \\
		\begin{small}
			\verb|openssl <rsa/ec> -text -noout -in <entrada> -out <salida>|
		\end{small}

		Adicionalmente, quizás necesitemos incluir \textit{-pubin} si la entrada es una clave pública o
		\textit{-passin <password>} si la entrada es una clave encriptada.

\section{Análisis de los resultados}
	\subsection{Valores de las claves de RSA}
		El contenido del archivo \textit{<nombre>RSAkey.txt} es:
		\lstinputlisting{./Resultados/MyRSAkey.txt}
		
		Aquí podemos ver que se almacenan algunos valores particulares para agilizar su uso en aplicaciones posteriores
		\footnote{Esta información ha sido sacada de \href{https://en.wikipedia.org/wiki/RSA\_(cryptosystem)\#Using\_the\_Chinese\_remainder\_algorithm}
		{RSA\_(cryptosystem) - Wikipedia}  con wxMaxima mediante el uso de \textit{is(<x> = <y>)}, \textit{mod(<x>, <y>)},
		\textit{ibase:<n>}, \textit{obase:<n>}, transformando los valores hexadecimales a decimal y almacenándolos en varibles.}:
		\begin{itemize}
			\item \textbf{Prime1}: que es uno de nuestros primos, $p$.
			\item \textbf{Prime2}: que es el otro de nuestros primos, $q$.
			\item \textbf{Modulus}: que es el producto de ambos primos, $n = p \cdot q$.
			\item \textbf{PublicExponent}: el exponente público, $3 \le e \le \varphi(n)$, que nos permite cifrar: $c \equiv_n m^e$.
			Además, hacer uso de 3, 17 ó $2^{16}+1$ como es nuestro caso, acelera las cuentas del cifrado por el \textbf{Método
			de los Cuadrados Iterados}.
			\item \textbf{PrivateExponent}: el inverso del exponente público, $d \equiv_{\varphi(n)} e^{-1}$, que nos
			permite descifrar: $m \equiv_n (m^e)^{e^{-1}}\equiv_n c^d$.
			\item \textbf{Exponent1}: exponente simplificado, $d_P =_{p-1} d$, nos permite agilizar las cuentas del descifrado
			usando el \textbf{Teorema Chino del Resto}.
			\item \textbf{Exponent2}: exponente simplificado, $d_Q =_{q-1} d$, nos permite agilizar las cuentas del descifrado
			usando el \textbf{Teorema Chino del Resto}.
			\item \textbf{Coefficient}: $q_{inv} =_{p} q^{-1}$, nos permite agilizar las cuentas del descifrado usando el
			\textbf{Teorema Chino del Resto}.
		\end{itemize}
		
		También podemos ver su tamaño, en este caso, 768 bits.
		
		El contenido del archivo \textit{<nombre>RSApriv.txt} es:
		\lstinputlisting{./Resultados/MyRSApriv.txt}
		
		Podemos apreciar que tiene los mismos valores que el anterior, esto es porque realmente no extraemos la clave privada,
		sólo la hemos cifrado.
		
		Finalmente, el contenido del archivo \textit{<nombre>RSApub.txt} es:
		\lstinputlisting{./Resultados/MyRSApub.txt}
		
		Aquí nos damos cuenta de que, como era de esperar, los únicos valores que se transmiten al público son \textbf{Modulus},
		\textbf{PublicExponent} y, por supuesto, su tamaño.
		
	\subsection{Cifrando y descrifrando RSA}
		En este apartado, haremos uso de otra función de \href{http://manpages.ubuntu.com/manpages/zesty/en/man1/openssl.1ssl.html}
		{\textit{openssl}}, en particular de \href{https://www.openssl.org/docs/man1.0.2/apps/rsautl.html}{\textit{rsautl}} con
		los parámetros \textit{encrypt} y \textit{decrypt}.
	
		Lo primero que nos viene a la cabeza cuando tenemos un criptosistema es utilizarlo directamente para cifrar nuestro
		mensaje usando: \\
		\begin{small}
			\verb|openssl rsautl -encrypt -pubin -inkey $name"RSApub.pem" -in input.bin -out input_RSA.bin|
		\end{small}	
		
		Y éste es el error que nos encontramos\footnote{He tenido que reducir la letra para poder mostrarlo en una sola línea.}:
		\lstinputlisting[basicstyle=\tiny]{./Resultados/error.txt}
		
		Este error viene a decirnos que nuestro mensaje es demasiado grande para ser cifrado usando la clave RSA que hemos
		generado. Así que lo que haremos será cifrarlo usando un cifrado por bloques con unos parámetros específicos y serán
		estos parámetros los que pasaremos cifrados por RSA:
		
		\begin{small}
			\verb|openssl rsautl -encrypt -pubin -inkey $name"RSApub.pem" -in sessionkey -out skey.c|
			\verb|openssl enc $mode -pass file:sessionkey -in input.bin -out output.bin|
		\end{small}
		
		Análogamente, se actúa para descifrarlo:
		
		\begin{small}
			\verb|openssl rsautl -decrypt -passin <pass> -inkey $name"RSApriv.pem" -in skey.c -out skey.d| \\
			\verb!dec_mode=`cat skey.d | tail -n 1`! \\
			\verb|openssl enc -d $dec_mode -pass file:skey.d -in output.bin -out output_d.bin|
		\end{small}
		
		Finalmente, sólo queda comprobar que los archivos efectivamente son idénticos.
		
		sessionkey:
		\lstinputlisting{./Resultados/sessionkey}
		
		skey.d:
		\lstinputlisting{./Resultados/sessionkey.dec}
		
		input.txt:
		\lstinputlisting{./Resultados/input.txt}
		
		output\_d.txt:
		\lstinputlisting{./Resultados/output_dec.txt}
	
	\subsection{Valores de las claves de Curvas Elípticas}
		El contenido del archivo \textit{<nombre>ECkey.txt} es:
		\lstinputlisting{./Resultados/MyECkey.txt}
		
		Aquí podemos ver que únicamente se almacenan los valores privados y públicos junto al nombre de la curva. También
		podemos ver su tamaño, en este caso, 192 bits.
		
		El contenido del archivo \textit{<nombre>ECpriv.txt} es:
		\lstinputlisting{./Resultados/MyECpriv.txt}
		
		Podemos apreciar que tiene los mismos valores que el anterior, esto es porque realmente no extraemos la clave privada,
		sólo la hemos cifrado.
		
		Finalmente, el contenido del archivo \textit{<nombre>ECpub.txt} es:
		\lstinputlisting{./Resultados/MyECpub.txt}
		
		Aquí nos damos cuenta de que, como era de esperar, el único valor que se transmite al público es \textbf{pub} junto
		con, obviamente, su tamaño y el nombre de la curva.
			
		\lstinputlisting{./Resultados/MyECpub.txt}

\end{document}
