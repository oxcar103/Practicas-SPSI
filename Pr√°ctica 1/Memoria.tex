%%
% Plantilla de Memoria
% Modificación de una plantilla de Latex de Nicolas Diaz para adaptarla 
% al castellano y a las necesidades de escribir informática y matemáticas.
%
% Editada por: Mario Román
%
% License:
% CC BY-NC-SA 3.0 (http://creativecommons.org/licenses/by-nc-sa/3.0/)
%%

%%%%%%%%%%%%%%%%%%%%%
% Thin Sectioned Essay
% LaTeX Template
% Version 1.0 (3/8/13)
%
% This template has been downloaded from:
% http://www.LaTeXTemplates.com
%
% Original Author:
% Nicolas Diaz (nsdiaz@uc.cl) with extensive modifications by:
% Vel (vel@latextemplates.com)
%
% License:
% CC BY-NC-SA 3.0 (http://creativecommons.org/licenses/by-nc-sa/3.0/)
%
%%%%%%%%%%%%%%%%%%%%%

%----------------------------------------------------------------------------------------
%	PAQUETES Y CONFIGURACIÓN DEL DOCUMENTO
%----------------------------------------------------------------------------------------

%% Configuración del papel.
% microtype: Tipografía.
% mathpazo: Usa la fuente Palatino.
\documentclass[a4paper, 11pt]{article}
\usepackage[protrusion=true,expansion=true]{microtype}
\usepackage{mathpazo}

% Indentación de párrafos para Palatino
\setlength{\parindent}{0pt}
  \parskip=8pt
\linespread{1.05} % Change line spacing here, Palatino benefits from a slight increase by default


%% Castellano.
% noquoting: Permite uso de comillas no españolas.
% lcroman: Permite la enumeración con numerales romanos en minúscula.
% fontenc: Usa la fuente completa para que pueda copiarse correctamente del pdf.
\usepackage[spanish,es-noquoting,es-lcroman]{babel}
\usepackage[utf8]{inputenc}
\usepackage[T1]{fontenc}
\selectlanguage{spanish}

%% Gráficos
\usepackage{graphicx} % Required for including pictures
\usepackage{wrapfig} % Allows in-line images
\usepackage[usenames,dvipsnames]{color} % Coloring code

%% Matemáticas
\usepackage{amsmath}

\makeatletter

% Hipervínculos
\usepackage[hidelinks]{hyperref}

%----------------------------------------------------------------------------------------
%	TÍTULO
%----------------------------------------------------------------------------------------
% Configuraciones para el título.
% El título no debe editarse aquí.
\renewcommand{\maketitle}{
  \begin{flushright} % Right align
  
  {\LARGE\@title} % Increase the font size of the title
  
  \vspace{50pt} % Some vertical space between the title and author name
  
  {\large\@author} % Author name
  \\\@date % Date
  \vspace{40pt} % Some vertical space between the author block and abstract
  \end{flushright}
}

% Título
\title{\textbf{Seguridad y Protección de Sistemas Informáticos (SPSI)}\\ % Title
Criptosistemas Simétricos} % Subtitle

\author{\textsc{Óscar Bermúdez Garrido} % Author
\\{\textit{Universidad de Granada}}} % Institution

\date{\today} % Date


%----------------------------------------------------------------------------------------
%	DOCUMENTO
%----------------------------------------------------------------------------------------

\begin{document}

\maketitle % Print the title section

% Resumen (Descomentar para usarlo)
\renewcommand{\abstractname}{Resumen} % Uncomment to change the name of the abstract to something else

% Índice
{\parskip=2pt
  \tableofcontents
}
\pagebreak

%% Inicio del documento

\section{Creación de los archivos}
	\subsection{Input.bin e Input1.bin}
		Para la creación de input.bin, utilizaremos el comando \href{http://manpages.ubuntu.com/manpages/zesty/en/man1/dd.1.html}
		{\textit{dd}} con entrada \textit{/dev/zero} que es un directorio que sólo aporta 0's. Para la creación de input1.bin
		podríamos usar \href{http://manpages.ubuntu.com/manpages/zesty/en/man1/ghex.1.html}{\textit{ghex}} para modificar
		input.bin y convertirlo en input1.bin añadiendo un 1 en la posición adecuada pero en este caso he usado el comando
		\href{http://manpages.ubuntu.com/manpages/zesty/en/man1/cat.1.html}{\textit{cat}} para poder encargárselo a un script.
		
	\subsection{Claves y vector de inicialización}
		Es fácil encontrar \href{https://en.wikipedia.org/wiki/Weak_key#Weak_keys_in_DES}{claves débiles de DES}. En nuestro
		script, hemos optado por usar \textbf{0x0101010101010101} como clave débil y \textbf{0x011F011F010E010E} como clave
		semidébil.
		
		Para el caso de la clave fuerte, nos basta tomar una clave hexadecimal que no esté en la lista de claves débiles
		mencionadas anteriormente. Optaremos por la clave \textbf{0x2103456789abcdef} % Esta permutación puede parecer casual pero no lo es... (103)
		para el cifrado de los ejercicios 4 en adelante. Y como vector de inicialización hemos usado \textbf{0x67}. % 103=0x67

	\subsection{Resultados}
		Para generar los resultados, nos basta con usar \href{http://manpages.ubuntu.com/manpages/zesty/en/man1/openssl.1ssl.html}
		{\textit{openssl}}:
		
		\begin{small}
			\verb|openssl enc -$mode -K $key -iv $vector -in input.bin -out ./Resultados/input_$mode.bin|
			\verb|openssl enc -$mode -K $key -iv $vector -in input1.bin -out ./Resultados/input1_$mode.bin|
		\end{small}
		
		Englobado todo en un \href{http://manpages.ubuntu.com/manpages/zesty/en/man3/for.3tcl.html}{\textit{for}} para ir
		pasando los parámetros adecuados en cada caso.
		
		La única excepción sería cuando nos piden desencriptar, pero para realizarla, nos basta con usar el parámetro \textit{-d}.
	
\section{Comparación y análisis de los resultados}
	\subsection{Modos ECB, CBC y OFB de DES con claves débil y semi-débil}
		En la encriptación por el modo ECB se trabaja por bloques de 64 bits de forma independiente\footnote{El algoritmo del
		modo ECB sería: \[c_i := e_k(m_i) \text{ para cifrar}\] \[m_i := d_k(c_i) \text{ para descifrar}\] con $e_k$ y $d_k$
		las funciones de cifrado y descrifado respectivamente generadas por la clave $k$, $c_i$ el $i$-ésimo criptograma, y
		$m_i$ el $i$-ésimo bloque. \label{ecb}}, por lo que podemos ver que tanto la clave débil como la semi-débil (y también
		la fuerte como veremos más adelante) hacen que el primer bloque sea diferente entre	\textit{input1\_des-ecb\_weak.bin}
		y \textit{input\_des-ecb\_weak.bin}, y entre \textit{input1\_des-ecb\_semiweak.bin} y \textit{input\_des-ecb\_semiweak.bin}
		pues \textit{input1.bin} y \textit{input.bin} sólo difieren en el primer bloque.
		
		En la encriptación por el modo CBC se trabaja por bloques de 64 bits combinados con el criptograma anterior mediante la
		operación xor ($\oplus$)\footnote{El algoritmo del modo CBC sería: \[c_i := e_k(m_i \oplus c_{i-1})\text{ para cifrar}\]
		\[m_i := d_k(c_i) \oplus c_{i-1} \text{ para descifrar}\] con $e_k$ y $d_k$ las funciones de cifrado y descrifado
		respectivamente generadas por la clave $k$, $c_i$ el $i$-ésimo criptograma con $c_0$ el vector de inicialización, y $m_i$
		el $i$-ésimo bloque. \label{cbc}}, por lo que podemos ver que en la clave débil, por tener todas las claves de ronda
		iguales, que los bloques de 64 bits pares\footnote{Entendamos como bloques de 64 bits pares aquellos que comprenden
		los bits 65-128, 193-256, 321-384 \dots, es decir, desde el bit $(2n-1) \cdot 64$ hasta el bit $2n \cdot 64$ con $n
		\in \mathbb{N}^*$. \label{bits-parity}} presentan el vector de inicialización junto con el primer bloque descifrado
		pues como el resto de bloques son $0's$ no lo alteran, y que todas las líneas de \textit{input1\_des-ecb\_weak.bin}
		y \textit{input\_des-ecb\_weak.bin} se repiten.

		En la encriptación por el modo OFB se trabaja por bloques de 64 bits combinados con el cifrado $(n-1)$-ésimo del vector
		de inicialización mediante la operación xor ($\oplus$)\footnote{El algoritmo del modo OFB sería: \[c_i := m_i \oplus
		e_k(s_{i-1}) \text{ para cifrar}\] \[m_i := c_i \oplus e_k(s_{i-1}) \text{ para descifrar}\] con $s_i = e_k(s_{i-1})$
		con $s_0$ el vector de inicialización, $e_k \equiv d_k$ la función de cifrado y descifrado generada por la clave $k$,
		$c_i$ el $i$-ésimo criptograma, y $m_i$ el $i$-ésimo bloque. \label{ofb}}, por lo que podemos ver que en la clave débil,
		por tener todas las claves de ronda iguales, que los bloques de 64 bits pares$^{\ref{bits-parity}}$ presentan el vector
		de inicialización y que todas los archivos \textit{input1\_des-ecb\_weak.bin} y \textit{input\_des-ecb\_weak.bin}, y
		\textit{input1\_des-ecb\_semiweak.bin} y \textit{input\_des-ecb\_semiweak.bin} son idéticos salvo el primer bloque
		pues \textit{input1.bin} y \textit{input.bin} sólo difieren en él.
				
	\subsection{Modo ECB de DES con clave fuerte}
		En este modo de encriptación al trabajar por bloques de 64 bits de forma independiente$^{\ref{ecb}}$, el hecho de trabajar
		con una clave fuerte no impide que la única diferencia entre \textit{input1\_des-ecb.bin} y \textit{input\_des-ecb.bin}
		sea el primer bloque pues es la única diferencia entre \textit{input1.bin} y \textit{input.bin}.
		
	\subsection{Modo CBC de DES con clave fuerte}
		En este modo de encriptación al trabajar por bloques de 64 bits combinados con el criptograma anterior mediante
		la operación xor ($\oplus$)$^{\ref{cbc}}$, el hecho de trabajar con una clave fuerte rompe la debilidad que tenían las
		claves débiles de que las claves de ronda se repetían y hace que \textit{input1\_des-cbc.bin} y \textit{input\_des-cbc.bin}
		sean completamente diferentes a pesar de haber una diferencia tan leve entre \textit{input1.bin} y \textit{input.bin}.
		
	\subsection{Modo ECB de AES-128 y AES-256 con clave fuerte}\label{AES-ECB}
		Este modo de encriptación es similar al del DES salvo que cambia el tamaño de la clave (en lugar de 64 bit como en
		DES es de 128 bits para AES-128 y de 256 bits para AES-256) y el tamaño de bloque, es decir, trabajar por bloques
		de 128 bits (frente a los 64 bits de DES) de forma independiente$^{\ref{ecb}}$. Al trabajar con bloques independientes
		vuelve a repetirse el problema del DES, y es que \textit{input1\_aes-128-ecb.bin} y \textit{input\_aes-128-ecb.bin}
		al igual que \textit{input1\_aes-256-ecb.bin} y \textit{input\_aes-256-ecb.bin} sólo difieren en el primer bloque
		pues es donde reside la única diferencia entre \textit{input1.bin} y \textit{input.bin} pero en este caso el bloque
		es de 128 bits.
		
	\subsection{Modo CBC de AES-128 y AES-256 con clave fuerte}\label{AES-CBC}
		Este modo de encriptación es similar al del DES salvo que cambia el tamaño de la clave (en lugar de 64 bit como en
		DES es de 128 bits para AES-128 y de 256 bits para AES-256) y el tamaño de bloque, es decir, trabajar por bloques
		de 128 bits (frente a los 64 bits de DES) combinándolos con el criptograma generado anteriormente mediante $\oplus$
		$^{\ref{cbc}}$. De esta forma, al trabajar con una clave fuerte logramos que sean totalmente diferentes tanto
		\textit{input1\_aes-128-cbc.bin} y \textit{input\_aes-128-cbc.bin} como \textit{input1\_aes-256-ecb.bin} y
		\textit{input\_aes-256-ecb.bin} a pesar de la similitud entre \textit{input1.bin} y \textit{input.bin}.
		
	\subsection{Modo OFB de AES-192 con clave fuerte}
		Este modo de encriptación es similar al del DES salvo que cambia el tamaño de la clave (en lugar de 64 bit como en
		DES es de 128 bits para AES-128 y de 256 bits para AES-256) y el tamaño de bloque, es decir, trabajar por bloques
		de 128 bits (frente a los 64 bits de DES) combinados con el cifrado $(n-1)$-ésimo del vector de inicialización
		mediante $\oplus$ $^{\ref{ofb}}$.
		
		Claramente, si tomamos la misma clave $k$ y el mismo vector de inicialización $s_0$, podemos observar que:
		\[m_i = c_i \oplus e_k(s_{i-1}) = [m_i \oplus e_k(s_{i-1})] \oplus e_k(s_{i-1}) = m_i\]
		
		Y básicamente eso es lo que podemos observar en \textit{output\_d\_aes-192-ofb.bin} y \textit{output\_e\_aes-192-ofb.bin}
		que, como no podría ser de otro modo, son además iguales que \textit{input.bin}
		
	\subsection{Otros cifrados simétricos: Camellia}
		Camellia es un algoritmo de cifrado simétrico de bloques desarrollado en Japón por Mitsubishi Electric y la NTT
		\footnote{Nippon Telegraph and Telephone Corporation.} of Japan con 128 bits de tamaño de bloque y varios tamaños de
		clave: 128, 192 ó 256 bits.
		
		Camellia usa cifrado de Feistel de 18 ó 24 rondas según el tamaño de la clave, y cada 6 rondas, se aplica una
		transformación lógica. Además, usa 4 S-cajas de 8$\times$8 bits con transformaciones afines de entrada y salida así
		como operaciones lógicas, junto con \textit{key whitening}\footnote{\textit{Key whitening} es una técnica orientada
		a incrementar la seguridad de un cifrado por bloques iterado, dificultando así el ataque por fuerza bruta. Consiste
		en combinar parcialmente el mensaje con la clave y la forma más usual de hacerlo es usar la operación xor antes de
		la primera ronda de encriptación y otra vez después de la última.}.

		Se considera que Camellia es imposible de romper por ataques de fuerza bruta sobre las claves (incluso sobre las de
		128 bits) con la tecnología actual y tampoco hay ataques conocidos que puedan debilitarlo suficientemente como para
		aplicar después otros ataques y lograr romperlo. Sus niveles de seguridad y su velocidad de procesado son comparables
		al cifrado AES de forma que su uso ha sido aprobado por \textbf{CRYPTREC} (Japón), \textbf{NESSIE} (Unión Europea)
		e \textbf{ISO} (\textit{International Organization for Standardization}).
		
		En cuanto a los resultados de la experimentación con \textit{openssl}, obtenemos los mismos resultados que obtuvimos
		con AES tanto en modo \hyperref[AES-ECB]{ECB} como en \hyperref[AES-CBC]{CBC}:
		\begin{itemize}
			\item \textbf{ECB}: Al trabajar con bloques independientes, podemos observar que los pares de archivos
			\textit{input1\_camellia-128-ecb.bin} y \textit{input\_camellia-128-ecb.bin}, \textit{input1\_camellia-192-ecb.bin}
			y \textit{input\_camellia-192-ecb.bin}, y por último \textit{input1\_camellia-256-ecb.bin} y
			\textit{input\_camellia-256-ecb.bin} sólo difieren en el primer bloque de 128 bits pues es el tamaño de bloque y
			la única diferencia entre \textit{input1.bin} y \textit{input.bin} reside al principio de ellos.
			
			\item \textbf{CBC}: Debido a la vinculación entre el último mensaje cifrado y el siguiente mensaje a cifrar, se
			logra que los pares \textit{input1\_camellia-128-cbc.bin} y \textit{input\_camellia-128-cbc.bin},
			\textit{input1\_camellia-192-cbc.bin} y \textit{input\_camellia-192-cbc.bin}, y finalmente \textit{input1\_camellia-256-cbc.bin}
			y \textit{input\_camellia-256-cbc.bin} sean totalmente diferentes a pesar de la similitud entre \textit{input1.bin}
			y \textit{input.bin}.
		\end{itemize}
\end{document}
