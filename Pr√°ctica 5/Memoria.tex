%%
% Plantilla de Memoria
% Modificación de una plantilla de Latex de Nicolas Diaz para adaptarla 
% al castellano y a las necesidades de escribir informática y matemáticas.
%
% Editada por: Mario Román
%
% License:
% CC BY-NC-SA 3.0 (http://creativecommons.org/licenses/by-nc-sa/3.0/)
%%

%%%%%%%%%%%%%%%%%%%%%
% Thin Sectioned Essay
% LaTeX Template
% Version 1.0 (3/8/13)
%
% This template has been downloaded from:
% http://www.LaTeXTemplates.com
%
% Original Author:
% Nicolas Diaz (nsdiaz@uc.cl) with extensive modifications by:
% Vel (vel@latextemplates.com)
%
% License:
% CC BY-NC-SA 3.0 (http://creativecommons.org/licenses/by-nc-sa/3.0/)
%
%%%%%%%%%%%%%%%%%%%%%

%----------------------------------------------------------------------------------------
%	PAQUETES Y CONFIGURACIÓN DEL DOCUMENTO
%----------------------------------------------------------------------------------------

%% Configuración del papel.
% microtype: Tipografía.
% mathpazo: Usa la fuente Palatino.
\documentclass[a4paper, 11pt]{article}
\usepackage[protrusion=true,expansion=true]{microtype}
\usepackage{mathpazo}

% Indentación de párrafos para Palatino
\setlength{\parindent}{0pt}
  \parskip=8pt
\linespread{1.05} % Change line spacing here, Palatino benefits from a slight increase by default


%% Castellano.
% noquoting: Permite uso de comillas no españolas.
% lcroman: Permite la enumeración con numerales romanos en minúscula.
% fontenc: Usa la fuente completa para que pueda copiarse correctamente del pdf.
\usepackage[spanish,es-noquoting,es-lcroman]{babel}
\usepackage[utf8]{inputenc}
\usepackage[T1]{fontenc}
\selectlanguage{spanish}

%% Gráficos
\usepackage{graphicx} % Required for including pictures
\usepackage{wrapfig} % Allows in-line images
\usepackage[usenames,dvipsnames]{color} % Coloring code

%% Matemáticas
\usepackage{amsmath}

% Para algoritmos
\usepackage{algorithm}
\usepackage{algorithmic}
\usepackage{amsthm}
\floatname{algorithm}{Algoritmo}
\renewcommand{\listalgorithmname}{Lista de algoritmos}
\renewcommand{\algorithmicrequire}{\textbf{Entrada:}}
\renewcommand{\algorithmicensure}{\textbf{Salida:}}
\renewcommand{\algorithmicend}{\textbf{Fin}}
\renewcommand{\algorithmicif}{\textbf{Si}}
\renewcommand{\algorithmicthen}{\textbf{Entonces}}
\renewcommand{\algorithmicelse}{\textbf{En otro caso}}
\renewcommand{\algorithmicelsif}{\algorithmicelse,\ \algorithmicif}
\renewcommand{\algorithmicendif}{\algorithmicend\ \algorithmicif}
\renewcommand{\algorithmicfor}{\textbf{Para }}
\renewcommand{\algorithmicforall}{\textbf{Para cada}}
\renewcommand{\algorithmicdo}{\textbf{}}
\renewcommand{\algorithmicendfor}{\algorithmicend\ \algorithmicfor}
\renewcommand{\algorithmicwhile}{\textbf{Mientras}}
\renewcommand{\algorithmicendwhile}{\algorithmicend\ \algorithmicwhile}
\renewcommand{\algorithmicloop}{\textbf{Repetir}}
\renewcommand{\algorithmicendloop}{\algorithmicend\ \algorithmicloop}
\renewcommand{\algorithmicrepeat}{\textbf{Repetir}}
\renewcommand{\algorithmicuntil}{\textbf{Hasta que}}
\renewcommand{\algorithmicprint}{\textbf{Imprimir}} 
\renewcommand{\algorithmicreturn}{\textbf{Devolver}} 
\renewcommand{\algorithmictrue}{\textbf{Verdadero }} 
\renewcommand{\algorithmicfalse}{\textbf{Falso }} 
\renewcommand{\algorithmicand}{\textbf{Y}}
\renewcommand{\algorithmicor}{\textbf{O}}
\renewcommand{\algorithmicnot}{\textbf{No}}

\makeatletter

% Hipervínculos
\usepackage[hidelinks]{hyperref}

%% Para incluir archivos en texto plano
\usepackage{listings}

% Para incluir archivos de código
\usepackage{verbatim}

%----------------------------------------------------------------------------------------
%	TÍTULO
%----------------------------------------------------------------------------------------
% Configuraciones para el título.
% El título no debe editarse aquí.
\renewcommand{\maketitle}{
  \begin{flushright} % Right align
  
  {\LARGE\@title} % Increase the font size of the title
  
  \vspace{50pt} % Some vertical space between the title and author name
  
  {\large\@author} % Author name
  \\\@date % Date
  \vspace{40pt} % Some vertical space between the author block and abstract
  \end{flushright}
}

% Título
\title{\textbf{Seguridad y Protección de Sistemas Informáticos (SPSI)}\\ % Title
Puzzles Hash} % Subtitle

\author{\textsc{Óscar Bermúdez Garrido} % Author
\\{\textit{Universidad de Granada}}} % Institution

\date{\today} % Date


%----------------------------------------------------------------------------------------
%	DOCUMENTO
%----------------------------------------------------------------------------------------

\begin{document}

\maketitle % Print the title section

% Resumen (Descomentar para usarlo)
\renewcommand{\abstractname}{Resumen} % Uncomment to change the name of the abstract to something else

% Índice
{\parskip=2pt
  \tableofcontents
}
\pagebreak

%% Inicio del documento

\section{Creación de los archivos}
	\subsection{functions.sh}
		El contenido de este archivo es:

		\begin{scriptsize}
		\verbatiminput{functions.sh}
		\end{scriptsize}

		Procedamos a ver detalladamente su funcionamiento en los siguientes apartados:

		\subsubsection{Funcionamiento general}
			\begin{algorithm}[H]
				\begin{algorithmic}[1]
					\REQUIRE \ \\
						\texttt{text}, texto de entrada \\
						\texttt{output\_file}, fichero de salida \\ \
					\STATE{\texttt{Calculamos \$nonce}}
					\STATE{\texttt{id = \$text\$nonce}}
					\STATE{\texttt{cont} = 1}
					\STATE{\texttt{x} = \texttt{Valor\_Aleatorio()}}
					\STATE{\texttt{Hash(\$id\$x)}}
					\STATE{\texttt{value=valid\_Hash()}}
					\WHILE{\texttt{\$value != true}}
						\STATE{\texttt{cont} = \texttt{\$cont + 1}}
						\STATE{\texttt{x} = \texttt{Siguiente\_Valor()}}
						\STATE{\texttt{Hash(\$id\$x)}}
						\STATE{\texttt{value=valid\_Hash()}}
					\ENDWHILE
					\PRINT{\texttt{\$id, \$x, \$Hash, \$cont} > \texttt{\$output\_file}}
				\end{algorithmic}
				\caption{Funcionamiento General}
				\label{General}
			\end{algorithm}
	
		\subsubsection{Parámetros}
			Como parámetros a nuestro programa, podemos pasarle:
			\begin{itemize}
				\item \textbf{text} que será el texto de entrada a partir del cuál se generará el \textit{id}, por defecto
				tiene configurado un valor de "Texto de prueba".

				\item \textbf{b} que nos indica el número deseado de 0's que habrá en la cadena resultado, por defecto está
				fijada a "103", pero como eso podría tardar cerca de una semana, se recomiena cambiar este parámetro.

				\item \textbf{output1} que es el primer archivo de salida y se usará para el método aleatorio, por defecto la 
				salida será"random.csv".
				
				\item \textbf{output2} que es el archivo de salida que se usará para el método lineal, por defecto la salida
				será "linear.csv".
			\end{itemize}
			
			Están pensados para ser incluidos en ese orden de tal forma que puedes mandar \textbf{output1} si has mandados
			también \textbf{text} y \textbf{b} pero no afectará a lo que ocurra con \textbf{output2}.
			
			Se puede también llamar sin parámetros pero el valor por defecto de \textbf{b} es tan grande que no se debería.
			
		\subsubsection{Creación de la máscara}
			Dado que estamos trabajando con números hexadecimales enormes, el SO es incapaz de trabajar con ellos, pues la
			forma fácil de fabricar la máscara sería hacer una máscara de \textbf{F}'s del tamaño que queramos la final y
			después hacerle un desplazamiento a la izquierda y un AND a nivel de bits tal que: \\
			\verb|($mask << $b) & $mask|
			
			Sin embargo, nos vemos obligados a trabajar sólo con unos cuántos bits $n$ para que el SO pueda trabajar con ellos
			así que la opción pasa por calcular la máscara parte a parte. Esto nos lleva a diferenciar entre 3 casos entre
			la variable contador $n \cdot i$ y el parámetro \textbf{b} mencionado anteriormente:
			\begin{itemize}
				\item $n \cdot i < \textbf{b}$: en este caso, simplemente van a quedarnos $n$ 0's pues el desplazamiento es
				mayor que el límite superior de intervalo $[n \cdot i, n \cdot (i-1)]$.
				\item $n \cdot (i-1) > \textbf{b}$: en este caso, ocurre justo lo contrario que en la situación anterior,
				dándonos lugar a $n$ F's.
				\item $n \cdot i < \textbf{b} < n \cdot i$: este caso es el más interesante pues es de verdad donde se
				calcula parte de la máscara. Además, la forma nueva de hacerlo es muy similar a la fórmula mostrada anteriormente:
				\verb|($mask << $(i-b)) & $mask|
			\end{itemize}
			
		\subsubsection{Validación del Hash}
		\subsubsection{Nuevo candidato aleatorio}
		\subsubsection{Nuevo candidato lineal}
	\subsection{scripth.sh}
	
		\subsubsection{Funcionamiento general}
		\subsubsection{Parámetros}
		\subsubsection{Bucle de cálculo de valores}
		\subsubsection{Bucle de creación de tablas}

	\subsection{Cambios menores hechos a mano}

\section{Análisis de los resultados}
	\subsection{Resultados de la búsqueda aleatoria}
	\subsection{Resultados de la búsqueda lineal}
	\subsection{Tendencia del valor de la media}

\end{document}
