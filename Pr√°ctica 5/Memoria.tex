%%
% Plantilla de Memoria
% Modificación de una plantilla de Latex de Nicolas Diaz para adaptarla 
% al castellano y a las necesidades de escribir informática y matemáticas.
%
% Editada por: Mario Román
%
% License:
% CC BY-NC-SA 3.0 (http://creativecommons.org/licenses/by-nc-sa/3.0/)
%%

%%%%%%%%%%%%%%%%%%%%%
% Thin Sectioned Essay
% LaTeX Template
% Version 1.0 (3/8/13)
%
% This template has been downloaded from:
% http://www.LaTeXTemplates.com
%
% Original Author:
% Nicolas Diaz (nsdiaz@uc.cl) with extensive modifications by:
% Vel (vel@latextemplates.com)
%
% License:
% CC BY-NC-SA 3.0 (http://creativecommons.org/licenses/by-nc-sa/3.0/)
%
%%%%%%%%%%%%%%%%%%%%%

%----------------------------------------------------------------------------------------
%	PAQUETES Y CONFIGURACIÓN DEL DOCUMENTO
%----------------------------------------------------------------------------------------

%% Configuración del papel.
% microtype: Tipografía.
% mathpazo: Usa la fuente Palatino.
\documentclass[a4paper, 11pt]{article}
\usepackage[protrusion=true,expansion=true]{microtype}
\usepackage{mathpazo}

% Indentación de párrafos para Palatino
\setlength{\parindent}{0pt}
  \parskip=8pt
\linespread{1.05} % Change line spacing here, Palatino benefits from a slight increase by default


%% Castellano.
% noquoting: Permite uso de comillas no españolas.
% lcroman: Permite la enumeración con numerales romanos en minúscula.
% fontenc: Usa la fuente completa para que pueda copiarse correctamente del pdf.
\usepackage[spanish,es-noquoting,es-lcroman]{babel}
\usepackage[utf8]{inputenc}
\usepackage[T1]{fontenc}
\selectlanguage{spanish}

%% Gráficos
\usepackage{graphicx} % Required for including pictures
\usepackage{wrapfig} % Allows in-line images
\usepackage[usenames,dvipsnames]{color} % Coloring code

%% Matemáticas
\usepackage{amsmath}

% Para algoritmos
\usepackage{algorithm}
\usepackage{algorithmic}
\usepackage{amsthm}
\floatname{algorithm}{Algoritmo}
\renewcommand{\listalgorithmname}{Lista de algoritmos}
\renewcommand{\algorithmicrequire}{\textbf{Entrada:}}
\renewcommand{\algorithmicensure}{\textbf{Salida:}}
\renewcommand{\algorithmicend}{\textbf{Fin}}
\renewcommand{\algorithmicif}{\textbf{Si}}
\renewcommand{\algorithmicthen}{\textbf{Entonces}}
\renewcommand{\algorithmicelse}{\textbf{En otro caso}}
\renewcommand{\algorithmicelsif}{\algorithmicelse,\ \algorithmicif}
\renewcommand{\algorithmicendif}{\algorithmicend\ \algorithmicif}
\renewcommand{\algorithmicfor}{\textbf{Para }}
\renewcommand{\algorithmicforall}{\textbf{Para cada}}
\renewcommand{\algorithmicdo}{\textbf{}}
\renewcommand{\algorithmicendfor}{\algorithmicend\ \algorithmicfor}
\renewcommand{\algorithmicwhile}{\textbf{Mientras}}
\renewcommand{\algorithmicendwhile}{\algorithmicend\ \algorithmicwhile}
\renewcommand{\algorithmicloop}{\textbf{Repetir}}
\renewcommand{\algorithmicendloop}{\algorithmicend\ \algorithmicloop}
\renewcommand{\algorithmicrepeat}{\textbf{Repetir}}
\renewcommand{\algorithmicuntil}{\textbf{Hasta que}}
\renewcommand{\algorithmicprint}{\textbf{Imprimir}} 
\renewcommand{\algorithmicreturn}{\textbf{Devolver}} 
\renewcommand{\algorithmictrue}{\textbf{Verdadero }} 
\renewcommand{\algorithmicfalse}{\textbf{Falso }} 
\renewcommand{\algorithmicand}{\textbf{Y}}
\renewcommand{\algorithmicor}{\textbf{O}}
\renewcommand{\algorithmicnot}{\textbf{No}}

\makeatletter

% Hipervínculos
\usepackage[hidelinks]{hyperref}

%% Para incluir archivos en texto plano
\usepackage{listings}

% Para incluir archivos de código
\usepackage{verbatim}

%----------------------------------------------------------------------------------------
%	TÍTULO
%----------------------------------------------------------------------------------------
% Configuraciones para el título.
% El título no debe editarse aquí.
\renewcommand{\maketitle}{
  \begin{flushright} % Right align
  
  {\LARGE\@title} % Increase the font size of the title
  
  \vspace{50pt} % Some vertical space between the title and author name
  
  {\large\@author} % Author name
  \\\@date % Date
  \vspace{40pt} % Some vertical space between the author block and abstract
  \end{flushright}
}

% Título
\title{\textbf{Seguridad y Protección de Sistemas Informáticos (SPSI)}\\ % Title
Puzzles Hash} % Subtitle

\author{\textsc{Óscar Bermúdez Garrido} % Author
\\{\textit{Universidad de Granada}}} % Institution

\date{\today} % Date


%----------------------------------------------------------------------------------------
%	DOCUMENTO
%----------------------------------------------------------------------------------------

\begin{document}

\maketitle % Print the title section

% Resumen (Descomentar para usarlo)
\renewcommand{\abstractname}{Resumen} % Uncomment to change the name of the abstract to something else

% Índice
{\parskip=2pt
  \tableofcontents
}
\pagebreak

%% Inicio del documento

\section{Creación de los archivos}
	\subsection{functions.sh}
		El contenido de este archivo es:

		\begin{scriptsize}
		\verbatiminput{functions.sh}
		\end{scriptsize}

		Procedamos a ver detalladamente su funcionamiento en los siguientes apartados:

		\subsubsection{Funcionamiento general}
			\begin{algorithm}[H]
				\begin{algorithmic}[1]
					\REQUIRE \ \\
						\texttt{text}, texto de entrada \\
						\texttt{output\_file}, fichero de salida \\ \
					\STATE{\texttt{Calculamos \$nonce}}
					\STATE{\texttt{id = \$text\$nonce}}
					\STATE{\texttt{cont} = 1}
					\STATE{\texttt{x} = \texttt{Valor\_Aleatorio()}}
					\STATE{\texttt{Hash(\$id\$x)}}
					\STATE{\texttt{value=valid\_Hash()}}
					\WHILE{\texttt{\$value != true}}
						\STATE{\texttt{cont} = \texttt{\$cont + 1}}
						\STATE{\texttt{x} = \texttt{Siguiente\_Valor()}}
						\STATE{\texttt{Hash(\$id\$x)}}
						\STATE{\texttt{value=valid\_Hash()}}
					\ENDWHILE
					\PRINT{\texttt{\$id, \$x, \$Hash, \$cont} > \texttt{\$output\_file}}
				\end{algorithmic}
				\caption{Funcionamiento General}
				\label{General}
			\end{algorithm}
	
		\subsubsection{Parámetros}
			Como parámetros a nuestro programa, podemos pasarle:
			\begin{itemize}
				\item \textbf{text} que será el texto de entrada a partir del cuál se generará el \textit{id}, por defecto
				tiene configurado un valor de \textit{Texto de prueba}.

				\item \textbf{b} que nos indica el número deseado de 0's que habrá en la cadena resultado, por defecto está
				fijada a \textit{103}, pero como eso podría tardar cerca de una semana, se recomiena cambiar este parámetro.

				\item \textbf{output1} que es el primer archivo de salida y se usará para el método aleatorio, por defecto la 
				salida será \textit{random.csv}.
				
				\item \textbf{output2} que es el archivo de salida que se usará para el método lineal, por defecto la salida
				será \textit{linear.csv}.
			\end{itemize}
			
			Están pensados para ser incluidos en ese orden de tal forma que puedes mandar \textbf{output1} si has mandados
			también \textbf{text} y \textbf{b} pero no afectará a lo que ocurra con \textbf{output2}.
			
			Se puede también llamar sin parámetros pero el valor por defecto de \textbf{b} es tan grande que no se debería.
			
		\subsubsection{Creación de la máscara}
			Dado que estamos trabajando con números hexadecimales enormes, el SO es incapaz de trabajar con ellos, pues la
			forma fácil de fabricar la máscara sería hacer una máscara de \textbf{F}'s del tamaño que queramos la final y
			después hacerle un desplazamiento a la izquierda y un AND a nivel de bits tal que: \\
			\verb|($mask << $b) & $mask|
			
			Sin embargo, nos vemos obligados a trabajar sólo con unos cuántos bits $n$ para que el SO pueda trabajar con ellos
			así que la opción pasa por calcular la máscara parte a parte. Esto nos lleva a diferenciar entre 3 casos entre
			la variable contador $n \cdot i$ y el parámetro \textbf{b} mencionado anteriormente:
			\begin{itemize}
				\item $n \cdot i < \textbf{b}$: en este caso, simplemente van a quedarnos $n$ 0's pues el desplazamiento es
				mayor que el límite superior de intervalo $[n \cdot i, n \cdot (i-1)]$.
				\item $n \cdot (i-1) > \textbf{b}$: en este caso, ocurre justo lo contrario que en la situación anterior,
				dándonos lugar a $n$ F's.
				\item $n \cdot i < \textbf{b} < n \cdot i$: este caso es el más interesante pues es de verdad donde se
				calcula parte de la máscara. Además, la forma nueva de hacerlo es muy similar a la fórmula mostrada anteriormente:
				\verb|($mask << $(i-b)) & $mask|
			\end{itemize}
			
		\subsubsection{Validación del Hash}
			Para validar el hash, simplemente volvemos a cortar tanto la máscara como el hash a comprobar y lo comparamos
			trozo a trozo mediante AND de tal forma que el hash será válido si cada uno de sus trozos devuelve todo 0's al
			compararlo con la máscara.
			
			Esto se puede resumir visualmente en:
			\begin{algorithm}[H]
				\begin{algorithmic}[1]
					\REQUIRE \ \\
						\texttt{hash}, hash a comprobar \\
					\STATE{\texttt{value} = \TRUE}
					\FOR{$i < \#Hash/n$}
						\STATE{\texttt{submask} = \texttt{\$mask[$n \cdot i$, $n \cdot (i-1)$]}}
						\STATE{\texttt{subhash} = \texttt{\$hash[$n \cdot i$, $n \cdot (i-1)$]}}
						\IF{\texttt{\$submask \AND \$subhash} != 0}
							\STATE{\texttt{value} = \FALSE}
						\ENDIF
					\ENDFOR
					\RETURN{\texttt{\$value}}
				\end{algorithmic}
				\caption{Validación del Hash}
				\label{valid_Hash}
			\end{algorithm}
			
		\subsubsection{Nuevo candidato aleatorio}
			Para el nuevo candidato aleatorio, nos basta generar otro valor aleatorio del tamaño adecuado. En particular, usando
			la orden \textit{hexdump} con los flags \textit{-v} para que se muestre de forma ininterrumpida, \textit{-n} para
			indicarle el tamaño deseado de nuestro valor aleatorio, y \textit{-e} para darle formato.
		
		\subsubsection{Nuevo candidato lineal}
			La búsqueda lineal puede parecer que no tendría problema pero tiene el problema de que tenemos que tratar con
			hexadecimales demasiado grandes. Por tanto, hay que partir el valor en cachos, sumarle 1, comprobar si hay acarreo,
			compromar también que el valor que nos ha devuelto la suma tiene $n$ dígitos pues ignora los 0's a la izquierda.
			Esto puede no parecer una mala idea pero nosotros trabajamos con las partes del número, por lo que es necesario
			que si extraemos $n$ dígitos de una parte, vuelvan otros $n$, ni más ni menos nos vale.
		
	\subsection{scripth.sh}
		El contenido de este archivo es:

		\begin{scriptsize}
		\verbatiminput{script.sh}
		\end{scriptsize}
		
		Procedamos a ver detalladamente su funcionamiento en los siguientes apartados:
		
		\subsubsection{Funcionamiento general}
		\begin{algorithm}[H]
			\begin{algorithmic}[1]
				\REQUIRE \ \\
					\texttt{sample}, número de muestras \\
					\texttt{text}, texto de entrada \\
					\texttt{last}, último valor \\
					\texttt{first}, primer valor \\
					\texttt{increment}, incremento entre valores \\ \
					
				\FOR{$i$=\texttt{\$first}; $i$ < \texttt{\$last}; $i +=$ \texttt{\$increment} }
					\STATE{\texttt{Creamos los archivos de salida aleatoria (\$random\$i.csv) y lineal (\$linear\$i.csv)}}
					\STATE{\texttt{Creamos los archivos de salida aleatoria y lineal}}
					\FOR{$j$ < \texttt{\$sample}}
						\STATE{\texttt{./functions "\$text" \$i \$random\$i.csv \$linear\$i.csv}}
					\ENDFOR
				\ENDFOR
				\FOR{$i = $ \texttt{\{\$random, \$linear\}}}
					\PRINT{\texttt{\$Cabecera, \$Separador}}
					\FOR{$i$=\texttt{\$first}; $i$ < \texttt{\$last}; $i +=$ \texttt{\$increment} }
						\STATE{\texttt{sum}= \texttt{mean}=\texttt{elements}=0, \texttt{msg = "|"}}
						\STATE{\texttt{Creamos los archivos de salida aleatoria y lineal}}
						\FOR{$k \in$ \$i\$j.csv}
							\STATE{\texttt{msg+="$\$k$ $\backslash t$"|}}
							\STATE{\texttt{sum+=\$k}, \texttt{elements++}}
						\ENDFOR
						\STATE{\texttt{mean=\$sum/\$elements}}
						\STATE{\texttt{msg+="$\$sum$ $\backslash t$| $\$mean$ $\backslash t$|"}}
						\PRINT{\texttt{\$msg, \$Separador}}
					\ENDFOR
				\ENDFOR
			\end{algorithmic}
			\caption{Funciones del script}
			\label{Script}
		\end{algorithm}
		
		\subsubsection{Parámetros}
			Como parámetros a nuestro programa, podemos pasarle:
			\begin{itemize}
				\item \textbf{sample}
				
				\item \textbf{text} que será un parámetro que pasaremos directamente a ./functions.sh, por defecto tiene
				configurado \textit{Texto de prueba} al igual que lo tenía ./function.sh.

				\item \textbf{last}, \textbf{first} e \textbf{increment} regulan por defecto el último valor, el primero y
				el incremento del bucle for que controla qué valores le pasamos al parámetro \textbf{b} de ./functions.sh.
				Además, se piden en ese orden porque es el orden de preferencia para la orden \textit{seq}, una orden muy
				interesante para la construcción de bucles en bash. Por defecto, los valores asignados son \textit{103},
				\textit{1} y \textit{1}, respectivamente.
			\end{itemize}
			
			Y, al igual que ocurría con el caso de ./functions.sh, se recomiendo dar un valor al parámetro \textbf{last}
			porque si toma el de por defecto, el programa tardará muchísimo en terminar.
			
		\subsubsection{Bucle de cálculo de valores}
			En el bucle de cálculo de valores nos limitamos a crear con \textit{touch} los archivos de salida que pasaremos
			a functions.sh. Debido a que queremos guardar muestras de varios valores del parámetro \textit{b}, crearemos
			2 archivos por cada posible valor, diferenciando los archivos entre si fueron hechos por el método aleatorio
			o por el lineal, y además si fueron hechos con un valor de \textit{b} u otro.
			
			Adicionalmente, este bucle selecciona qué valores toma el parámetro \textit{b} a través de las variables
			\textbf{last}, \textbf{first} e \textbf{increment} y el comando \textit{seq} como comentábamos antes.
			
			Una vez establecidos, el valor de \textbf{b}, \textbf{output1} y \textbf{output2}, le dejamos el trabajo a
			functions.sh realizar su labor tantas veces como indique el parámetro \textbf{sample} por cada valor de
			\textbf{b}.
		
		\subsubsection{Bucle de creación de tablas}
			En el bucle de creación de tablas, agruparemos todos los archivos que fueron generados por el método aleatorio en
			un mismo archivo pero colocaremos en cada fila un archivo distinto. Análogamente, con los generados por el otro
			método.
			
			Además, este método se encarga también de imprimir la cabecera de la tabla, los separadores de filas y columnas,
			la suma y la media de los archivos para el cuál usaremos \textit{bc} con el flag \textit{-l} y el configurando
			"scale" al número de decimales que queramos.
			
			También cabe destacar el acceso a los valores de los archivos csv usando \textit{cut} y \textit{tr} dos grandes
			herramientas de procesamiento automático de textos dada su versatilidad y su sencillez, aunque como desventaja
			hay que decir también que requieren un poco de la imaginación del usuario de visualizar una solución aproximada
			primero para lograr aprovecharlas bien.

	\subsection{Cambios menores hechos a mano}
		Adicionalmente, he modificado momentáneamente script.sh para que muestre 1 cifra decimal para la media ya que no vamos a
		necesitar más pues tenemos 10 muestras pero es momentáneo porque quiero que el script sea lo más genérico posible.

		% Nótese que esto es una espinita clavada
		También, para una correcta visualización en este documento, se han tenido que modificar las tablas mediante la inserción
		de tabulaciones y lograr así que las columnas queden alineadas.
		
		En el caso de la tabla con más de 100 datos, se han eliminado también las columnas de los elementos pues hacían la tabla
		demasiado grande para visualizarla.

\section{Análisis de los resultados}
	\subsection{Resultados de la búsqueda aleatoria}
		La tabla con 10 muestras ha quedado así: \\
		\begin{scriptsize}
			\lstinputlisting{Tablas/random.txt}
		\end{scriptsize}
		
		Mientras que los resultados de la tabla con 103 muestras queda:
		\lstinputlisting{Tablas/random103.txt}
		
	\subsection{Resultados de la búsqueda lineal}
		La tabla ha quedado así: \\
		\begin{scriptsize}
			\lstinputlisting{Tablas/linear.txt}
		\end{scriptsize}
	
		Mientras que los resultados de la tabla con 103 muestras queda:
		\lstinputlisting{Tablas/linear103.txt}

	\subsection{Tendencia del valor de la media}
		Mediante combinatoria, podemos calcular muy fácilmente que la probabilidad de que un valor tenga un hash \textbf{b}
		0's es $\displaystyle \frac{1}{2^b}$. Por tanto, se deduce que en media necesitaremos $2^b$ intentos para lograr dar
		con un valor que cumpla estas características.
	
		Viendo las tablas anteriores, más o menos podemos ver cómo tienden a duplicarse el número de intentos medios entre
		un valor y otro, justo como nos indican los resultados teóricos. Sin embargo, no se ve de forma tan clara en todos
		los casos pues depende de una componente aleatoria y el número de muestras hace que la media sea bastante sensible
		a esta componente presente en ambos algoritmos.
		
		Como prueba de ello, podemos ver que que hay ocasiones en las que encuentra un valor cuya composición da un hash con
		8 ó 9 ceros en 6 ó 5 intentos respectivamente y sin embargo para encontrar un hash de 2 ceros requiere 31 intentos
		en una ocasión.
		Además, también podemos ver que la media es muy sensible pues en la tabla de búsqueda aleatoria, nos encontramos con
		que es más fácil encontrar en media una máscara con 14 ceros que con 13, cosa totalmente ilógica.
		
		Basándonos en la Ley de los Grandes Números, cuando repetimos el experimento con 103 muestras pero una \textbf{b} más
		limitada, podemos ver mejor la tendencia hacia este valor teórico.
		Aunque siga teniendo cierta diferencia con el valor teórico, la variación se reduce considerablemente y se reduciría
		aún más si tomamos un número de muestras mayor y suficiente paciencia como para dejarlo ejecutarse.
\end{document}
