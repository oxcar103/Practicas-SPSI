%%
% Plantilla de Memoria
% Modificación de una plantilla de Latex de Nicolas Diaz para adaptarla 
% al castellano y a las necesidades de escribir informática y matemáticas.
%
% Editada por: Mario Román
%
% License:
% CC BY-NC-SA 3.0 (http://creativecommons.org/licenses/by-nc-sa/3.0/)
%%

%%%%%%%%%%%%%%%%%%%%%
% Thin Sectioned Essay
% LaTeX Template
% Version 1.0 (3/8/13)
%
% This template has been downloaded from:
% http://www.LaTeXTemplates.com
%
% Original Author:
% Nicolas Diaz (nsdiaz@uc.cl) with extensive modifications by:
% Vel (vel@latextemplates.com)
%
% License:
% CC BY-NC-SA 3.0 (http://creativecommons.org/licenses/by-nc-sa/3.0/)
%
%%%%%%%%%%%%%%%%%%%%%

%----------------------------------------------------------------------------------------
%	PAQUETES Y CONFIGURACIÓN DEL DOCUMENTO
%----------------------------------------------------------------------------------------

%% Configuración del papel.
% microtype: Tipografía.
% mathpazo: Usa la fuente Palatino.
\documentclass[a4paper, 11pt]{article}
\usepackage[protrusion=true,expansion=true]{microtype}
\usepackage{mathpazo}

% Indentación de párrafos para Palatino
\setlength{\parindent}{0pt}
  \parskip=8pt
\linespread{1.05} % Change line spacing here, Palatino benefits from a slight increase by default


%% Castellano.
% noquoting: Permite uso de comillas no españolas.
% lcroman: Permite la enumeración con numerales romanos en minúscula.
% fontenc: Usa la fuente completa para que pueda copiarse correctamente del pdf.
\usepackage[spanish,es-noquoting,es-lcroman]{babel}
\usepackage[utf8]{inputenc}
\usepackage[T1]{fontenc}
\selectlanguage{spanish}

%% Gráficos
\usepackage{graphicx} % Required for including pictures
\usepackage{wrapfig} % Allows in-line images
\usepackage[usenames,dvipsnames]{color} % Coloring code

%% Matemáticas
\usepackage{amsmath}

\makeatletter

% Hipervínculos
\usepackage[hidelinks]{hyperref}

%% Para incluir archivos en texto plano
\usepackage{listings}

%----------------------------------------------------------------------------------------
%	TÍTULO
%----------------------------------------------------------------------------------------
% Configuraciones para el título.
% El título no debe editarse aquí.
\renewcommand{\maketitle}{
  \begin{flushright} % Right align
  
  {\LARGE\@title} % Increase the font size of the title
  
  \vspace{50pt} % Some vertical space between the title and author name
  
  {\large\@author} % Author name
  \\\@date % Date
  \vspace{40pt} % Some vertical space between the author block and abstract
  \end{flushright}
}

% Título
\title{\textbf{Seguridad y Protección de Sistemas Informáticos (SPSI)}\\ % Title
Protocolos Criptográficos} % Subtitle

\author{\textsc{Óscar Bermúdez Garrido} % Author
\\{\textit{Universidad de Granada}}} % Institution

\date{\today} % Date


%----------------------------------------------------------------------------------------
%	DOCUMENTO
%----------------------------------------------------------------------------------------

\begin{document}

\maketitle % Print the title section

% Resumen (Descomentar para usarlo)
\renewcommand{\abstractname}{Resumen} % Uncomment to change the name of the abstract to something else

% Índice
{\parskip=2pt
  \tableofcontents
}
\pagebreak

%% Inicio del documento

\section{Creación de los archivos}
	\subsection{Claves DSA}
		Para generar las claves DSA, primero necesitaremos crear los parámetros comunes de las mismas. Para ello, nos basta
		con usar \href{http://manpages.ubuntu.com/manpages/zesty/en/man1/openssl.1ssl.html}{\textit{openssl}} con la opción
		\href{https://www.openssl.org/docs/man1.0.2/apps/dsaparam.html}{\textit{dsaparam}} indicándole el tamaño en bits del
		primo $p$ que queramos:
		\begin{small}
			\verb|openssl dsaparam -out "sharedDSA.pem" <prime_size>|
		\end{small}

		Una vez tenemos estos parámetros, operamos análogamente a como hicimos para generar las claves RSA, con
		\href{https://www.openssl.org/docs/man1.0.2/apps/gendsa.html}{\textit{gendsa}} generamos la clave a partir de los
		parámetros y usamos \href{https://www.openssl.org/docs/man1.0.2/apps/dsa.html}{\textit{dsa}} para trabajar con ella,
		por ejemplo, para cifrarla o para extraer la clave pública de la privada. La principal diferencia es que tenemos que
		generar 2 claves:

		\begin{small}
			\verb|for i in $name $surname| \\
			\verb|    do| \\
			\verb|        openssl gendsa -out $i"DSAkey.pem" sharedDSA.pem| \\
			\verb|        openssl dsa -aes128 -passout <pass> -in $i"DSAkey.pem" -out $i"DSApriv.pem"| \\
			\verb|        openssl dsa -pubout -in $i"DSAkey.pem" -out $i"DSApub.pem"| \\
			\verb|    done| \\
		\end{small}
		
	\subsection{Valores Hash}
		Para la generación de archivos hash, ya sea firmados o no, en openssl utilizaremos el comando
		\href{https://www.openssl.org/docs/man1.0.2/apps/dgst.html}{\textit{dgst}} y dado que en los 3 ejemplos
		se utilizará el formato hexadecimal, deberíamos incluir el parámetro \textit{-hex} pero no lo haremos
		porque es el formato por defecto. Adicionalmente, incluiremos el parámetro \textit{-c} en todos ellos
		para mostrarlos agrupados en bloques de 2 separados por \textbf{:} (aunque en la práctica nos pide mostrar
		así obligatoriamente sólo el primero). También debemos pasarle por parámetro el algoritmo que queremos usar
		
		El primer ejemplo es el más fácil pues ya nos indican qué algoritmo usar y sólo hay que buscar el parámetro
		\textit{-c}:\\
		\begin{small}
			\verb|openssl dgst -sha384 -c -out $name"DSApub.sha384" $name"DSApub.pem"|
		\end{small}
		
		El siguiente ejemplo nos pide una salida de 160 bits y sólo hay un algoritmo con ese nombre:\\
		\begin{small}
			\verb|openssl dgst -ripemd160 -c -out $surname"DSApub.ripemd160" $surname"DSApub.pem"|
		\end{small}
		
		Para finalizar, generamos el valor HMAC con la clave especificada:\\
		\begin{small}
			\verb|openssl dgst -hmac "12345" -c -out sharedDSA.hmac sharedDSA.pem|
		\end{small}

	\subsection{Protocolo Estación a Estación}
		En este protocolo, lo primero será generar las claves que, en nuestro caso, serán una clave por curva elíptica como
		en la práctica anterior para generar la clave derivada y una clave DSA como la explicada anteriormente para firmar.
		
		Este protocolo se compone de varios mensajes iniciales entre los usuarios implicados en la conversación, a los que
		llamaremos, como es usual, \textbf{Alice} y \textbf{Bob}:
		
		\begin{itemize}
			\item Alice:
			\begin{itemize}
				\item Pasa su clave: $c = g^a$
			\end{itemize}
			\item Bob:
			\begin{itemize}
				\item Lee la clave de Alice: $c$
				\item Calcula la clave compartida: $k = c^b = g^{ab}$
				\item Firma $(c || d)$: $s = sgn_B (c || d)$
				\item Pasa su clave y la firma cifrada: $(d || e_k(s))$
			\end{itemize}
			\item Alice:
			\begin{itemize}
				\item Lee la clave y la firma cifrada de Bob: $d = g^b, e_k(s)$
				\item Calcula la clave compartida: $k = d^a = g^{ab}$
				\item Descifra la firma: $d_k(e_k(s))$
				\item Verifica que el mensaje lo ha firmado Bob
				\item Firma $(d || c)$: $t = sgn_A (d || c)$
				\item Pasa su firma cifrada: $e_k(t)$
			\end{itemize}
			\item Bob:
			\begin{itemize}
				\item Lee la firma cifrada de Alice: $e_k(t)$
				\item Descifra la firma: $d_k(e_k(t))$
				\item Verifica que el mensaje lo ha firmado Alice
			\end{itemize}
		\end{itemize}
		
		Donde podemos ver 3 componentes principales del algoritmo: clave derivada, cifrado/descrifrado y firma/verificación.
		
		\subsubsection{Claves derivadas}
			Para la generación de esta clave, tendremos que tomar la clave pública que nos han pasado y combinarla con nuestra
			clave privada y así obtener una nueva clave común que pertenece sólo a nosotros.
			
			Para realizar esta tarea en openssl, usaremos \href{https://www.openssl.org/docs/man1.0.2/apps/pkeyutl.html}
			{\textit{pkeyutl}} con la opción \textit{-derive} con ambas contraseñas: \textit{-inkey} para nuestra clave privada
			y \textit{-peerkey} para la clave pública que nos han pasado.
			
			En concreto, quedaría algo así:
			\begin{small}
				\verb|openssl pkeyutl -derive -passin <pass> -inkey <nuestra clave>|
				\verb| -peerkey <clave recibida> -out <clave derivada>|
			\end{small}
				
		\subsubsection{Cifrar y Descrifrar}
			Realmente, esta parte ya la conocemos pues la hemos trabajado bastante a fondo en la primera parte y hemos seguido
			viéndola en uso.
			
			Como ya sabemos, se utiliza \href{https://www.openssl.org/docs/man1.0.2/apps/enc.html}{\textit{enc}} para esta
			tarea junto con el modo (que esta vez será \textit{-aes-128-cfb8}, el cuál no habíamos tenido que usar hasta
			ahora), la clave que derivamos antes, y el flag \textit{-d} si queremos desencriptar el mensaje:
			
			\begin{small}
				\verb|openssl enc (¿-d?) <mode> -pass <clave derivada> -in <entrada> -out <salida>|
			\end{small}
		
		\subsubsection{Firmar y Verificar}		
			Finalmente, procederemos a explicar el método de firma y verificación.
			
			Para este método, openssl trae la funcionalidad de \href{https://www.openssl.org/docs/man1.0.2/apps/dgst.html}
			{\textit{dgst}}. En particular, el método de firma y verificación son relativamente similares:

			\begin{small}
				\verb|openssl dgst <modo> -passin <pass> -sign <clave privada> -out <firma> <original>|
				\verb|openssl dgst <modo> -verify <clave pública> -signature <firma> <original>|
			\end{small}
			
			La principal diferencia está en que al firmar, utilizamos nuestra propia clave privada (\textit{-sign}) pero
			al verificar utilizamos la clave pública del emisor (\textit{-verify}). También cabe destacar que la función hash
			escogida ha sido \textit{-sha256}.
			
\section{Análisis de los resultados}
	\subsection{Claves DSA}
		El contenido del archivo \textit{sharedDSA} es:
		\lstinputlisting{./Resultados/sharedDSA.txt}

		En él podemos ver los valores de $P$, $Q$ y $G$, siendo $P$ y $Q$ dos primos con la relación $P = 2kQ+1$ con $k \in
		\mathbb{N}$ y $P > G = H^{\frac{P-1}{Q}} \mod{p} \neq 1$ con $h \in \mathbb{Z}^*_p$
		
		Al visualizar \textit{<nombre>DSApriv.txt} obtenemos:
		\lstinputlisting{./Resultados/MyDSApriv.txt}
		
		Mientras que con \textit{<apellido>DSApriv.txt} obtenemos:
		\lstinputlisting{./Resultados/OtherDSApriv.txt}
		
		Podemos apreciar que $P$, $Q$ y $G$ se mantienen pues son los valores heredados de \textit{sharedDSA}, y las claves
		pública y privada son distintas. Podemos apreciar además que la clave pública es mucho mayor que la privada. Esto se
		debe a su origen, pues mientras que la clave privada es un valor comprendido entre 1 y $Q-1$, ambos incluidos, la
		clave pública se genera elevando $G$ al valor de la clave privada y después empleando el módulo por $P$.

		Por su parte, las claves públicas sólo contienen los parámetros de generación junto con la parte pública como podemos
		comprobar en \textit{<nombre>DSApub.txt}:
		\lstinputlisting{./Resultados/MyDSApub.txt}
		
		Y en \textit{<apellido>DSApub.txt}:
		\lstinputlisting{./Resultados/OtherDSApub.txt}
		
	\subsection{Valores Hash}
		Tras usar sha384 sobre \textit{<nombre>DSApub.pem}, obtenemos:
		\lstinputlisting{./Resultados/MyDSApub.sha384}
		
		Análogamente, al aplicar ripem160 sobre \textit{<apellido>DSApub.pem} nos produce la siguiente salida de 160 bits:
		\lstinputlisting{./Resultados/OtherDSApub.ripemd160}
		
		Como último ejemplo, vemos que el valor HMAC de \textit{sharedDSA.pem} es:
		\lstinputlisting{./Resultados/sharedDSA.hmac}

	\subsection{Protocolo Estación a Estación}
		Cuando Bob firma (c||d), el resultado que nos encontramos es:
		\lstinputlisting{./Resultados/sgnB.txt}

		Sin embargo, como debe cifrarlo primero para mandárselo a Alice, lo que realmente manda es:
		\lstinputlisting{./Resultados/sgnB_enc.txt}
		
		Después, Alice lo descifra y queda de la forma:
		\lstinputlisting{./Resultados/sgnB_dec.txt}
		
		Que como podemos comprobar, es idéntico a como era antes de cifrarlo, por lo que la verificación da positivo.
		
		A continuación, Alice firma (d||c) dando como resultado:
		\lstinputlisting{./Resultados/sgnA.txt}
		
		Pero, de nuevo, debe cifrarlo. Y finalmente el archivo que envía quedó así:
		\lstinputlisting{./Resultados/sgnA_enc.txt}
		
		Bob, al descifrarlo, obtiene:
		\lstinputlisting{./Resultados/sgnA_dec.txt}
		
		Que como vuelve a coincidir con el archivo sin cifrar, nos vuelve a dar una verificación positiva.
		
		Por tanto, Alice y Bob ahora tiene la clave derivada que fueron pasando y están completamente seguros de que los
		mensajes que reciban se los ha enviado el otro, sin intermediarios.
	
\end{document}
