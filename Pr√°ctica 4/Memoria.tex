%%
% Plantilla de Memoria
% Modificación de una plantilla de Latex de Nicolas Diaz para adaptarla 
% al castellano y a las necesidades de escribir informática y matemáticas.
%
% Editada por: Mario Román
%
% License:
% CC BY-NC-SA 3.0 (http://creativecommons.org/licenses/by-nc-sa/3.0/)
%%

%%%%%%%%%%%%%%%%%%%%%
% Thin Sectioned Essay
% LaTeX Template
% Version 1.0 (3/8/13)
%
% This template has been downloaded from:
% http://www.LaTeXTemplates.com
%
% Original Author:
% Nicolas Diaz (nsdiaz@uc.cl) with extensive modifications by:
% Vel (vel@latextemplates.com)
%
% License:
% CC BY-NC-SA 3.0 (http://creativecommons.org/licenses/by-nc-sa/3.0/)
%
%%%%%%%%%%%%%%%%%%%%%

%----------------------------------------------------------------------------------------
%	PAQUETES Y CONFIGURACIÓN DEL DOCUMENTO
%----------------------------------------------------------------------------------------

%% Configuración del papel.
% microtype: Tipografía.
% mathpazo: Usa la fuente Palatino.
\documentclass[a4paper, 11pt]{article}
\usepackage[protrusion=true,expansion=true]{microtype}
\usepackage{mathpazo}

% Indentación de párrafos para Palatino
\setlength{\parindent}{0pt}
  \parskip=8pt
\linespread{1.05} % Change line spacing here, Palatino benefits from a slight increase by default


%% Castellano.
% noquoting: Permite uso de comillas no españolas.
% lcroman: Permite la enumeración con numerales romanos en minúscula.
% fontenc: Usa la fuente completa para que pueda copiarse correctamente del pdf.
\usepackage[spanish,es-noquoting,es-lcroman]{babel}
\usepackage[utf8]{inputenc}
\usepackage[T1]{fontenc}
\selectlanguage{spanish}

%% Gráficos
\usepackage{graphicx} % Required for including pictures
\usepackage{wrapfig} % Allows in-line images
\usepackage[usenames,dvipsnames]{color} % Coloring code

%% Matemáticas
\usepackage{amsmath}

\makeatletter

% Hipervínculos
\usepackage[hidelinks]{hyperref}

%% Para incluir archivos en texto plano
\usepackage{listings}

%----------------------------------------------------------------------------------------
%	TÍTULO
%----------------------------------------------------------------------------------------
% Configuraciones para el título.
% El título no debe editarse aquí.
\renewcommand{\maketitle}{
  \begin{flushright} % Right align
  
  {\LARGE\@title} % Increase the font size of the title
  
  \vspace{50pt} % Some vertical space between the title and author name
  
  {\large\@author} % Author name
  \\\@date % Date
  \vspace{40pt} % Some vertical space between the author block and abstract
  \end{flushright}
}

% Título
\title{\textbf{Seguridad y Protección de Sistemas Informáticos (SPSI)}\\ % Title
Certificados Digitales} % Subtitle

\author{\textsc{Óscar Bermúdez Garrido} % Author
\\{\textit{Universidad de Granada}}} % Institution

\date{\today} % Date


%----------------------------------------------------------------------------------------
%	DOCUMENTO
%----------------------------------------------------------------------------------------

\begin{document}

\maketitle % Print the title section

% Resumen (Descomentar para usarlo)
\renewcommand{\abstractname}{Resumen} % Uncomment to change the name of the abstract to something else

% Índice
{\parskip=2pt
  \tableofcontents
}
\pagebreak

%% Inicio del documento

\section{Creación de los archivos}
	\subsection{Archivo de configuración}
		\subsubsection{¿Qué es el archivo de configuración?}
			Por defecto, \href{http://manpages.ubuntu.com/manpages/zesty/en/man1/openssl.1ssl.html}{\textit{openssl}} usa el
			archivo de configuración ubicado en \textit{/usr/lib/ssl/openssl.cnf} pero podemos modificarlo teniendo en cuenta
			las consideraciones oportunas que nos indica \href{https://www.openssl.org/docs/man1.0.2/apps/config.html}
			{\textit{OpenSSL CONF library configuration files}} para que así \textit{openssl} no tenga ningún problema en
			interpretar la nueva configuración.
			
		\subsubsection{Modificación de la configuración}
			En mi caso particular, prefería mantener el archivo original, por lo que he hecho una copia del mismo en el
			directorio de trabajo y lo he modificado ahí. Esto me obliga a indicarle su ubicación a \textit{openssl} cada
			vez que lo necesite añadiendo el flag:\\
			\begin{small}
				\verb|-config my_openssl.cnf|\\
			\end{small}
		
			He modificado los siguientes campos:
			\begin{itemize}
				\item \textbf{dir}: que regula el directorio donde trabaja la \textit{Autoridad Certificadora(CA)} y le he fijado
				el valor a \textit{./Claves}.
				
				\item \textbf{unique\_subject}: que impide la repetición de los parámetros en los certificados. Este parámetro se
				debería dejar siempre a \textit{yes} o estar comentado (pues toma el valor \textit{yes} por defecto) dado que
				\textbf{DN} significa \textbf{Distinguible Name} y si los certificados son idénticos, se rompe el sentido de
				\textbf{DN}. Sin embargo, dado que esto es una práctica para nuestro aprendizaje sin utilidad real, he preferido
				omitir esto y repetir los parámetros pues no es de interés para la práctica saber crear valores distintos, así que
				lo he fijado a \textit{no}.
				
				\item  \textbf{default\_days}: que establece el tiempo por defecto de validez de los certificados emitidos si no
				aparece la flag \textit{-days}, lo he cambiado desde \textit{365} hasta \textit{30}.
			\end{itemize}
			
		\subsubsection{Otros cambios}
			Al igual que los parámetros que he modificado, me he encontrado otros parámetros en dicho fichero que he visto
			interesantes para modificarlos pero no he visto la necesidad de hacerlo. Estos son:
			\begin{itemize}
				\item Los subdirectorios y ficheros por defecto de CA:
				\begin{itemize}
					\item \textit{certificate} y \textit{private\_key} como los ficheros de la CA.
					
					\item \textit{certs} y \textit{new\_certs\_dir} como directorios para los Certificados.
					
					\item \textit{crl\_dir}, \textit{crlnumber} y \textit{crl} para la Lista de Certificados Revocados.
					
					\item \textit{database} y \textit{serial} que llevan el registro de los Certificados emitidos. Por tanto,
					es importante inicializarlos. En mi caso, ejecuto:\\
					\begin{small}
						\verb|echo "1000" > Claves/serial|\\
						\verb|touch ./Claves/index.txt|\\
					\end{small}
				\end{itemize}
				
				\item Los parámetros de identificación por defecto de un certificado y su longitud permitida:
				\begin{itemize}
					\item \textit{countryName}, \textit{stateOrProvinceName} y \textit{localityName} como parámetros para
					indicar la localización del propietario.
					
					\item \textit{0.organizationName}, \textit{1.organizationName} y \textit{organizationalUnitName} para
					la identificación de las organizaciones que acreditan el certificado.
					
					\item \textit{commonName} y \textit{emailAddress} como identificación personal del propietario.
				\end{itemize}
				
				\item \textbf{default\_bits}: Los bits de la clave RSA que se crea por defecto, está fijada a \textit{2048}.
			\end{itemize}
			
	\subsection{Autoridad Certificadora Raíz}
		Para generar la \textit{Autoridad Certificadora Raíz}(\textbf{Root CA}) usaremos \textit{openssl} con la opción
		\href{https://www.openssl.org/docs/man1.0.2/apps/req.html}{\textit{req}} con los flags:
		\begin{itemize}
			\item \textit{-new}, pues lo que queremos es hacer una nueva solicitud de certificado.
			\item \textit{-x509}, pues queremos que esté auto-firmado.
			\item \textit{-days}, pues queremos darle un periodo de validez superior al de los certificados que vamos a firmar.
			\item \textit{-keyout}, para indicarle donde queremos guardar la clave privada.
			\item \textit{-out}, para indicalre donde queremos guardar el certificado resultante.
		\end{itemize}
		
		Adicionalmente, para evitar que nos pregunte los parámetros de identificación personal, podemos pasárselos por línea de
		comandos mediante el flag \textit{-subj}.
		
		En particular, generamos nuestra CA raíz usando:\\
		\begin{small}
			\verb|openssl req -new -passout <pass> -x509 -days 103 -config my_openssl.cnf|\\
			\verb| -subj "<param>" -keyout Claves/private/cakey.pem -out Claves/cacert.pem|\\
		\end{small}
		
		También podemos hacerlo usando el script \href{https://www.openssl.org/docs/man1.0.2/apps/CA.pl.html}{\textit{CA.pl}}
		ubicado en la dirección \textit{/usr/lib/ssl/misc/CA.pl} mediante las órdenes:\\
		\begin{small}
			\verb|CA.pl -newca|\\
			\verb|CA.pl -newcert|\\
		\end{small}
	
	\subsection{Solicitud de Certificado con clave por defecto}
		Para la generación de Solicitudes de Certificados es obligatorio el manejo de la instrucción \textit{req} junto con
		sus parámetros, en particular, \textit{new}. Adicionalmente, aunque yo no lo he hecho, podemos usar \textit{-newkey}
		que nos permite seleccionar el tipo de encriptación así como los parámtros deseados. Por defecto, y es el protocolo
		que uso por comodidad, es el \textbf{RSA} con el tamaño en bits por defecto \textit{2048}, como ya comenté pues no
		modifiqué ese valor en el archivo de configuración.
		
		Quedaría algo similar a:\\
		\begin{small}
			\verb|openssl req -new -passout <pass> (-newkey <algorithm>:<conf>) -config my_openssl.cnf|\\
			\verb| -subj "<param>" -keyout Claves/private/default_key.pem -out Solicitudes/default_key.pem|\\
		\end{small}
	
	\subsection{Solicitud de Certificado de una clave existente}
		Para la generación de Solicitudes de Certificados con clave existente, previamente creamos una clave\footnote{Ver
		prácticas anteriores.}\footnote{En mi caso, he usado una clave DSA.} y después llamamos a \textit{req -new} como
		hicimos en el caso anterior sólo que ahora hay 2 pequeñas diferencias muy relacionadas entre sí:
		\begin{itemize}
			\item No necesitamos crear una clave, luego no necesitamos hacer uso de \textit{-newkey}.
			
			\item Queremos usar una clave existente, luego en lugar de usar \textit{-keyout} para indicar dónde guardar la
			nueva clave, usaremos \textit{-key} para indicar dónde está la clave que vamos a usar.
		\end{itemize}
		
		Finalmente, el resultado sería algo así:\\
		\begin{small}
			\verb|openssl req -new -passin <pass> -config my_openssl.cnf -subj "<param>"|\\
			\verb| -key Claves/private/DSApriv.pem -out Solicitudes/prev_key.pem|\\
		\end{small}
	
	\subsection{Certificación de Solicitudes}
		Para la Certificación de Solicitudes utilizaremos la opción \href{https://www.openssl.org/docs/man1.0.2/apps/ca.html}
		{\textit{ca}} de \textit{openssl} con los flags:
		\begin{itemize}
			\item \textit{-noemailDN}, para indicarle que, a pesar de que es un dato presente en la Solicitud de Certificado,
			este dato no aparezca en el Certificado que finalmente se firme por la \textit{CA}. Esto es una buena práctica
			según el propio manual.
			\item \textit{-in}, para indicarle donde está la solicitud que se quierer firmar.
			\item \textit{-out}, para indicalre donde queremos guardar el certificado resultante.
		\end{itemize}
		
		Adicionalmente, podemos pasarle un archivo \textit{CA} distinto al que se encuentra en la dirección por defecto dada
		por el archivo de configuración mediante la opción \textit{-cert}.
		
		De este modo, la orden para la certificación quedaría tal que:\\		
		\begin{small}
			\verb|openssl ca -batch -noemailDN -passin <pass> -config my_openssl.cnf|\\
			\verb| (-cert <CA file>) -in Solicitudes/<req> -out Claves/newcerts/<cert>|\\
		\end{small}

		El parámetro \textit{-batch} utilizado anteriormente, nos permite aceptar automáticamente las solicitudes que sean
		correctas, es decir, que tengan un \textbf{DN} distinto a los ya utilizados en el caso usual de que en el archivo de
		configuración \textbf{unique\_subject} no se haya establecido a \textit{no}\footnote{Como ya mencioné anteriormente,
		es importante que ese parámetro sea configurado de ese modo pues permite respetar la utilidad de \textbf{DN} pero
		esto no es más que una práctica para la familiarización con los comandos, de modo que no le ví sentido respetarlo
		durante la ejecución de la misma pero sí que comprendo la importancia que tiene.}. De nuevo, he vuelto a obviar la
		importancia de algo durante la ejecución de la práctica. En este caso, al eliminar la interacción en este punto estoy
		pasando por alto que los datos introducidos al solicitar un certificado sean correctos pues podría indicarse una
		ubicación inexistente, un correo o nombre que no existan por tener un formato incorrecto o similar. Simplemente lo
		hago por agilizar y porque como soy yo mismo el que introduce los parámetros, sé que estarán correctos\footnote{O
		al menos, eso creo, cualquier contribución será bienvenida. \textasciicircum-\textasciicircum}.

		También podemos hacerlo usando el anteriormente mencionado script \textit{CA.pl} mediante la orden:\\
		\begin{small}
			\verb|CA.pl -signreq|\\
		\end{small}
		
	\subsection{Archivos de valores}
		Para realizar el volcado de los datos en archivos para su posterior análisis se hará diferenciando casos:
		\begin{itemize}
			\item Para los Certificados usamos \href{https://www.openssl.org/docs/man1.0.2/apps/x509.html}{\textit{x509}}:\\
			\begin{small}
				\verb|openssl x509 -text -noout -in Claves/<cert>.pem > Resultados/c_<cert>.txt|\\
			\end{small}
			
			Cabe destacar que las principales funciones de este comando son precisamente la visualización y la firma de
			certificados. Por lo que tiene más funciones como mostrar un dato en particular, por ejemplo:
			\begin{itemize}
				\item \textbf{-email} para mostrar el correo electrónico asociado al certificado si hay alguno.
				\item \textbf{-startdate} para mostrar cuándo empieza el periodo de validez del certificado.
				\item \textbf{-enddate} para mostrar cuándo termina el periodo de validez del certificado.
				\item \textbf{-dates} equivale a \textbf{-startdate} y \textbf{-enddate} simultáneamente.
				\item \textbf{-pubkey} muestra la clave pública.
			\end{itemize}
			
			Para imprimirlos todos, utilizamos un bucle \textit{for} sobre el directorio \textit{Claves/newcerts} y mostramos
			también \textit{cacert.pem} que sería el único certificado que no está en dicho directorio.
			
			\item De manera análoga, para los Certificados usamos \href{https://www.openssl.org/docs/man1.0.2/apps/req.html}
			{\textit{req}} para mostrarlos:\\
			\begin{small}
				\verb|openssl req -text -noout -in Solicitudes/<req>.pem -out Resultados/r_<req>.txt|\\
			\end{small}
			
			Para imprimirlos todos, simplemente utilizamos un bucle \textit{for} sobre el directorio \textit{Solicitudes}.
		\end{itemize}
		
		Como se puede ver en la separación de casos, diferenciamos los archivos de salida incluyendo \textbf{r\_} si proviene
		de una solicitud o \textbf{c\_} si lo hace de un certificado. Esta diferenciación simplemente se hace para evitar
		sobreescritura pues cuando creamos un certificado a partir de una solicitud, no le cambiamos el nombre. Por ello,
		si volcáramos sin más el archivo, sólo se conservaría uno de los 2 (en particular, el último creado por el script
		que es el certificado). Adicionalmente, comentar que como el CA no tiene un equivalente en el directorio de
		Solicitudes, es el único que su volcado mantiene su nombre cambiando la extensión \textit{.pem} por \textit{.txt}.
		
\section{Análisis de los resultados}
	\subsection{Autoridad Certificadora Raíz}
		El archivo \textit{cacert.txt} nos muestra:
		\lstinputlisting[basicstyle=\footnotesize]{./Resultados/cacert.txt}
		
		Como podemos ver, la \textit{Autoridad Certificadora raíz} es un certificado autofirmado como mencionamos
		anteriormente y esto se refleja en que tanto \textbf{Issuer} que es la identidad que firma el certificado como
		\textbf{Subject} que es la identidad que lo solicita, coinciden. Otra forma de ver esto es en la parte de X509 que
		podemos ver 2 apartados interesantes más:
		\begin{itemize}
			\item En la parte donde pone \textit{Basic Constraints} podemos ver que se confirma que es la \textit{CA}
			mediante el texto \verb|CA:TRUE|
			
			\item Por otro lado, vemos que el identificador de clave de \textbf{Subject} es idéntico al de la \textit{CA}.
		\end{itemize}
		
		Además, también hemos dicho que su periodo de validez debe ser mayor que el resto para proveer suficiente tiempo
		de vigencia para que los certificados emitidos por ella tengan utilidad. Podemos ver que su tiempo de validez es
		superior a 3 veces el del resto de certificados como veremos más adelante cuando veamos el resto de archivos.
		
		Por último, destacaremos la inclusión de la información de la clave pública de \textbf{Subject} así como el
		algoritmo de firma que tiene.
		
	\subsection{Solicitudes de Certificados}
		En el caso de las solicitudes, la estructura es similar a la \textit{CA} pero con muchos menos campos pues sólo
		aparece la información del \textbf{Subject} (como no está firmada aún, no tiene \textbf{Issuer}), la clave pública
		y la firma.
		
		El archivo de la solicitud que genera la clave es:
		\lstinputlisting[basicstyle=\footnotesize]{./Resultados/r_default_key.txt}
		
		El archivo de la solicitud para una clave existente es:
		\lstinputlisting[basicstyle=\footnotesize]{./Resultados/r_prev_key.txt}
		
	\subsection{Certificación de Solicitudes}
		Antes de firmar un certificado, nos muestra la información referente al mismo para que comprobemos que es correcta,
		el periodo de validez que tendrá, qué configuración se va a usar, su número de serie y la información de X509. Una
		vez mostrada, nos pregunta si queremos firmarlo y si queremos documentarla en la base de datos.
		
		Para firmar la solicitud que genera la clave, se nos muestra:
		\lstinputlisting[basicstyle=\footnotesize]{./Resultados/default_key.txt}
		
		Una vez comprobada la información y firmada, obtenemos el certificado:
		\lstinputlisting[basicstyle=\footnotesize]{./Resultados/c_default_key.txt}
		
		Además, automáticamente se genera un archivo por defecto con la misma información en \textit{Claves/newcerts} cuyo
		nombre se corresponde con el número de serie del certificado:
		\lstinputlisting[basicstyle=\footnotesize]{./Resultados/c_1000.txt}
		
		Para firmar la solicitud para una clave existente, se nos muestra:
		\lstinputlisting[basicstyle=\footnotesize]{./Resultados/prev_key.txt}
		
		Una vez comprobada la información y firmada, obtenemos el certificado:
		\lstinputlisting[basicstyle=\footnotesize]{./Resultados/c_prev_key.txt}
		
		Y en este caso, el archivo generado de forma automática sería:
		\lstinputlisting[basicstyle=\footnotesize]{./Resultados/c_1001.txt}
		
		De nuevo, incidiendo en que la \textit{CA} es un certificado autofirmado, podemos ver que la estructura de los
		certificados es la misma que la mostrada en el apartado de \textit{CA}. Aparecen el campo de \textbf{Issuer} y
		\textbf{Subject}\footnote{Estos campos deben mantener igual la información de \textit{C(country)},
		\textit{ST(State or Province Name)} y \textit{O(Organization Name)} para que la \textit{CA} tenga autoridad pues no
		puede certificar solicitudes de fuera de su rango mientras que \textit{DN} debería ser distinto y único pero, aunque
		sí que es distinto entre \textbf{Issuer} y \textbf{Subject}, todos los solicitantes tienen los mismos datos y ya
		comenté que quitamos esta restricción junto con una justificación de por qué se tomó dicha decisión.} que nos dan
		información sobre la \textit{CA} y la identidad que realizó la solicitud, respectivamente; el periodo de validez,
		que es el periodo que fijamos por defecto; la clave pública y la firma del \textbf{Subject}; y el apartado de X509
		que nos muestra que el identificador de clave de \textbf{Subject} es distinto al de la \textit{CA} y en
		\textit{Basic Constraints} aparece el texto \verb|CA:FALSE|.
		
\end{document}
