%%
% Plantilla de Memoria
% Modificación de una plantilla de Latex de Nicolas Diaz para adaptarla 
% al castellano y a las necesidades de escribir informática y matemáticas.
%
% Editada por: Mario Román
%
% License:
% CC BY-NC-SA 3.0 (http://creativecommons.org/licenses/by-nc-sa/3.0/)
%%

%%%%%%%%%%%%%%%%%%%%%
% Thin Sectioned Essay
% LaTeX Template
% Version 1.0 (3/8/13)
%
% This template has been downloaded from:
% http://www.LaTeXTemplates.com
%
% Original Author:
% Nicolas Diaz (nsdiaz@uc.cl) with extensive modifications by:
% Vel (vel@latextemplates.com)
%
% License:
% CC BY-NC-SA 3.0 (http://creativecommons.org/licenses/by-nc-sa/3.0/)
%
%%%%%%%%%%%%%%%%%%%%%

%----------------------------------------------------------------------------------------
%	PAQUETES Y CONFIGURACIÓN DEL DOCUMENTO
%----------------------------------------------------------------------------------------

%% Configuración del papel.
% microtype: Tipografía.
% mathpazo: Usa la fuente Palatino.
\documentclass[a4paper, 11pt]{article}
\usepackage[protrusion=true,expansion=true]{microtype}
\usepackage{mathpazo}

% Indentación de párrafos para Palatino
\setlength{\parindent}{0pt}
  \parskip=8pt
\linespread{1.05} % Change line spacing here, Palatino benefits from a slight increase by default


%% Castellano.
% noquoting: Permite uso de comillas no españolas.
% lcroman: Permite la enumeración con numerales romanos en minúscula.
% fontenc: Usa la fuente completa para que pueda copiarse correctamente del pdf.
\usepackage[spanish,es-noquoting,es-lcroman]{babel}
\usepackage[utf8]{inputenc}
\usepackage[T1]{fontenc}
\selectlanguage{spanish}

%% Gráficos
\usepackage{graphicx} % Required for including pictures
\usepackage{wrapfig} % Allows in-line images
\usepackage[usenames,dvipsnames]{color} % Coloring code

%% Matemáticas
\usepackage{amsmath}

\makeatletter

% Hipervínculos
\usepackage[hidelinks]{hyperref}

%% Para incluir archivos en texto plano
\usepackage{listings}

%----------------------------------------------------------------------------------------
%	TÍTULO
%----------------------------------------------------------------------------------------
% Configuraciones para el título.
% El título no debe editarse aquí.
\renewcommand{\maketitle}{
  \begin{flushright} % Right align
  
  {\LARGE\@title} % Increase the font size of the title
  
  \vspace{50pt} % Some vertical space between the title and author name
  
  {\large\@author} % Author name
  \\\@date % Date
  \vspace{40pt} % Some vertical space between the author block and abstract
  \end{flushright}
}

% Título
\title{\textbf{Seguridad y Protección de Sistemas Informáticos (SPSI)}\\ % Title
Protocolos Criptográficos} % Subtitle

\author{\textsc{Óscar Bermúdez Garrido} % Author
\\{\textit{Universidad de Granada}}} % Institution

\date{\today} % Date


%----------------------------------------------------------------------------------------
%	DOCUMENTO
%----------------------------------------------------------------------------------------

\begin{document}

\maketitle % Print the title section

% Resumen (Descomentar para usarlo)
\renewcommand{\abstractname}{Resumen} % Uncomment to change the name of the abstract to something else

% Índice
{\parskip=2pt
  \tableofcontents
}
\pagebreak

%% Inicio del documento

\section{Creación de los archivos}
	\subsection{Archivo de configuración}
		\subsubsection{¿Qué es el archivo de configuración?}
			Por defecto, \href{http://manpages.ubuntu.com/manpages/zesty/en/man1/openssl.1ssl.html}{\textit{openssl}} usa el
			archivo de configuración ubicado en \textit{/usr/lib/ssl/openssl.cnf} pero podemos modificarlo teniendo en cuenta
			las consideraciones oportunas que nos indica \href{https://www.openssl.org/docs/man1.0.2/apps/config.html}
			{\textit{OpenSSL CONF library configuration files}} para que así \textit{openssl} no tenga ningún problema en
			interpretar la nueva configuración.
			
		\subsubsection{Modificación de la configuración}
			En mi caso particular, prefería mantener el archivo original, por lo que he hecho una copia del mismo en el
			directorio de trabajo y lo he modificado ahí. Esto me obliga a indicarle su ubicación a \textit{openssl} cada
			vez que lo necesite añadiendo el flag:\\
			\verb|-config my_openssl.cnf|\\
		
			He modificado los siguientes campos:
			\begin{itemize}
				\item \textbf{dir}: que regula el directorio donde trabaja la \textit{Autoridad Certificadora(CA)} y le he fijado
				el valor a \textit{./Claves}.
				
				\item \textbf{unique\_subject}: que impide la repetición de los parámetros en los certificados. Este parámetro se
				debería dejar siempre a \textit{yes} o estar comentado (pues toma el valor \textit{yes} por defecto) dado que
				\textbf{DN} significa \textbf{Distinguible Name} y si los certificados son idénticos, se rompe el sentido de
				\textbf{DN}. Sin embargo, dado que esto es una práctica para nuestro aprendizaje sin utilidad real, he preferido
				omitir esto y repetir los parámetros pues no es de interés para la práctica saber crear valores distintos, así que
				lo he fijado a \textit{no}.
				
				\item  \textbf{default\_days}: que establece el tiempo por defecto de validez de los certificados emitidos si no
				aparece la flag \textit{-days}, lo he cambiado desde \textit{365} hasta \textit{30}.
			\end{itemize}
			
		\subsubsection{Otros cambios}
			Al igual que los parámetros que he modificado, me he encontrado otros parámetros en dicho fichero que he visto
			interesantes para modificarlos pero no he visto la necesidad de hacerlo. Estos son:
			\begin{itemize}
				\item Los subdirectorios y ficheros por defecto de CA:
				\begin{itemize}
					\item \textit{certificate} y \textit{private\_key} como los ficheros de la CA.
					
					\item \textit{certs} y \textit{new\_certs\_dir} como directorios para los Certificados.
					
					\item \textit{crl\_dir}, \textit{crlnumber} y \textit{crl} para la Lista de Certificados Revocados.
					
					\item \textit{database} y \textit{serial} que llevan el registro de los Certificados emitidos. Por tanto,
					es importante inicializarlos. En mi caso, ejecuto:\\
					\verb|echo "1000" > Claves/serial|\\
					\verb|touch ./Claves/index.txt|\\
				\end{itemize}
				
				\item Los parámetros de identificación por defecto de un certificado y su longitud permitida:
				\begin{itemize}
					\item \textit{countryName}, \textit{stateOrProvinceName} y \textit{localityName} como parámetros para
					indicar la localización del propietario.
					
					\item \textit{0.organizationName}, \textit{1.organizationName} y \textit{organizationalUnitName} para
					la identificación de las organizaciones que acreditan el certificado.
					
					\item \textit{commonName} y \textit{emailAddress} como identificación personal del propietario.
				\end{itemize}
				
				\item \textbf{default\_bits}: Los bits de la clave RSA que se crea por defecto, está fijada a \textit{2048}.
			\end{itemize}
			
	\subsection{Autoridad Certificadora Raíz}
	\subsection{Solicitud de Certificado con clave por defecto}
	\subsection{Solicitud de Certificado de una clave existente}
	\subsection{Certificación de Solicitudes}
	\subsection{Archivos de valores}
			
\section{Análisis de los resultados}
	\subsection{Autoridad Certificadora Raíz}
	\subsection{Solicitudes de Certificados}
	\subsection{Certificación de Solicitudes}
	
\end{document}
